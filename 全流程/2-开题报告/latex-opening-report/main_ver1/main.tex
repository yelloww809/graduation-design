%%=============================================
% !Mode:: "TeX:UTF-8"
% !TEX program  = XeLaTeX
%%=============================================
% 模板名称:hitszthesis
% 模板版本:V3.4
% 模板作者:杨敬轩(Jingxuan Yang)
% 联系作者:yangjx20@mails.tsinghua.edu.cn & yanglatex2e@gmail.com
% 模板交流:QQ群:1039392552,加群请备注LaTeX、hitszthesis相关说明
% 模板适用:哈尔滨工业大学(深圳)本、硕、博学位论文
% 模板编译:手动编译方法参看 README.md 或 hitszthesis.pdf
%          GNU make 工具:make thesis
%          Windows 批处理脚本:双击 compile.bat 自动编译论文
%          更多编译细节详见说明文档:hitszthesis.pdf
% 更新时间:2024/11/22
% 模板帮助:请**务必务必务必**阅读 hitszthesis.pdf 说明文档,文档查看方法:
%          cmd 命令行:texdoc hitszthesis
%          推荐前往模板的 GitHub 仓库获取最新文件,地址:
%          https://github.com/YangLaTeX/hitszthesis
%%=============================================

% 设置文档类别为 <hitszthesis>
% \documentclass[type=doctor]{hitszthesis}
% \documentclass[type=master]{hitszthesis}
\documentclass[type=bachelor,infoleft=true,phase=proposal,isessay=true]{hitszthesis}
% \documentclass[type=bachelor,infoleft=true,phase=interim,isessay=true]{hitszthesis}

% 模板提供以下选项,各个选项之间不要有空格
% 新1. 选择毕业设计或论文
% isessay=true|false,true 为论文,false 为设计,默认为 true
% 新2. 选择为开题或中期
% phase=proposal|interim,默认为结题论文时不需增加此项
% 1. type=bachelor|master|doctor
%   含义:本科、硕士、博士学位论文,不设默认值,**必填**
% 2. covertitletworow=true|false
%   含义:本科封面第一页标题单行或多行显示,默认为单行显示(false)
% 3. infoleft=true|false
%   含义:本科封面第二页下划线内容居中或居左显示,默认为居中显示(false)
% 4. mathfont=newtxmath|mtprotwolite|mtprotwo
%   含义:正文数学字体选项:newtxmath(默认),mtprotwolite(lite版,免费),
%         mtprotwo(完全版,需购买授权),
%         mtpro2字体官网:https://www.pctex.com/mtpro2.html
% 5. boldcaption=true|false
%   含义:图表题注是否加粗,默认为不加粗(false)
% 6. tocfour=true|false
%   含义:是否添加第四级目录,只对本科文科个别要求四级目录有效,默认不添加(false)
% 7. fulltime=true|false
%   含义:是否全日制,非全日制如同等学力等,要在coverinformation中设置类型,
%        默认是全日制(true)
% 8. subtitle=true|false
%   含义:论文题目是否含有副标题,默认没有副标题(false)
% 9. openright=true|false
%   含义:博士论文是否要求章节首页必须在奇数页,默认否(false)
% 10. library=true|false
%   含义:是否为提交到图书馆的电子版,默认否(false)
% 11. alphappendix=true|false
%   含义:本科毕业设计附录章节编号是否为大写字母,默认是(true)
% 12. bsfrontpagenumberline=true|false
%   含义:本科毕业设计前言部分页码是否添加短横线,默认是(true)
% 13. bsmainpagenumberline=true|false
%   含义:本科毕业设计正文部分页码是否添加短横线,默认是(true)

% 自定义设置与额外加载的宏包请写在 \file{hitszthesis.sty} 里
\usepackage{hitszthesis}

% 图片存放路径,在这些文件夹里的图片可以直接使用图片文件名调用
\graphicspath{{figures/}{pictures/}}

%%=============================================
% 开始写论文
% !!注意本文仅作为排版格式示例,并不作为毕业论文规范
\begin{document}

% 开始写前言部分
\frontmatter

% 封面信息填写
% !TEX root = ../main.tex

\hitszsetup{
  %******************************
  % 注意:
  %   1. 配置里面不要出现空行
  %   2. 不需要的配置信息可以删除
  %******************************
  %=====
  % 秘级
  %=====
  statesecrets={公开},
  natclassifiedindex={TM301.2},
  intclassifiedindex={62-5},
  %=========
  % 中文信息
  %=========
  ctitleone={基于深度学习的无线电信号},%本科生封面使用
  ctitletwo={时频定位与制式识别方法研究},%本科生封面使用
  ctitlecover={基于深度学习的无线电信号时频定位与制式识别方法研究},%放在封面中使用,自由断行
  ctitle={基于深度学习的无线电信号时频定位与制式识别方法研究},%放在原创性声明中使用
  csubject={通信工程},
  caffil={信息科学与技术学院},
  cauthor={黄文浩},
  csupervisor={张行健 副教授},
  cdate={2025年10月13日},      % 日期自动使用当前时间,若需指定按如下方式修改
  cdatesecond={2025年10月13日},     % 指定第二页封面的日期,即答辩日期
  cstudentid={220210809},
}


% 生成封面、中英文摘要
\makecover

% 物理量名称表,若采用标准符号则不需要此表
% \input{front/denotation}

% 中文目录
\tableofcontents

% 英文目录,本硕不要求
% \tableofengcontents

% 开始写正文
\mainmatter

% 以下为开题的章节内容

% !TEX root = ../main.tex

\chapter{课题背景及研究的目的与意义}

% ========== 1.1 课题背景 ==========
\section{课题背景}

随着物联网(Internet of Things, IoT)技术的迅猛发展,无线通信设备数量正呈爆炸式增长。
从智能家居、工业自动化到智慧医疗与无人驾驶,海量的无线设备与通信技术在有限的频谱资源中交织共存,构成了日益拥挤且动态多变的复杂电磁环境。
在此背景下,信号检测(Signal Detection)技术的重要性日益凸显,它不仅是提升频谱共享效率、降低信号间干扰的关键,也是保障无线通信安全、监控非法信号的必要手段。

与仅关注频段是否被占用的传统频谱感知(Spectrum Sensing, SS)不同,
信号检测旨在同时完成两项核心子任务:
(1)对环境中的信号进行精确的时频定位(即确定信号在何时占用哪些频谱资源);
(2)对定位到的信号进行准确的制式识别(即判断信号的调制类型或通信协议)。
由此可知,信号检测是实现动态频谱接入(Dynamic Spectrum Access, DSA)和保障通信可靠性的关键前提。

然而,能量检测、匹配滤波的传统频段定位方式和基于机器学习的传统信号分类方式由于其对噪声功率敏感、依赖信号先验信息以及计算复杂度高等固有局限,
难以完全适应当前非协作、高动态的复杂电磁环境。

近年来,深度学习(Deep Learning, DL)领域的突破性进展为应对上述挑战提供了全新思路。
通过将一维时域信号经过短时傅里叶变换(Short-Time Fourier Transform, STFT)转化为二维时频图,
信号感知任务得以巧妙地转化为计算机视觉领域中技术成熟的目标检测(Object Detection)问题。
这种跨域迁移方法的优势在于,
它能够借助目标检测模型实现端到端的处理,在无需精确信号或信道先验知识的条件下,自动学习信号在时频域的深层特征,从而在信号的精确定位与制式分类方面展现出巨大潜力。

% ========== 1.2 研究的目的与意义 ==========
\section{研究的目的与意义}

尽管计算机视觉领域的目标检测技术已日趋成熟,并涌现出众多先进方案,
但其在信号检测领域的应用仍处于初步探索阶段,不仅相关研究成果有限,且现有的技术迁移模式也存在若干亟待解决的问题:

1. 数据集与真实场景存在偏差:
目前,该领域的研究大多采用仿真生成或混合真实信号的数据集。
这些数据集普遍存在对现实场景的过度简化,例如信号制式与调制类型有限、信号时频密度较低,且通常忽略了信号间的重叠与干扰。
此外,现有数据集普遍假设了固定的采样率与采样时长,这与实际监测中参数多变的情况存在显著差异。
虽然这种简化有助于在研究初期快速验证算法的可行性,但却使其难以准确评估模型在真实复杂电磁环境下的性能与泛化能力。

2. 模型迁移缺乏针对性优化:
现有的研究在模型迁移方面,大多直接套用计算机视觉领域的经典模型(如Faster R-CNN),而未根据信号时频图的内在特征(相较于自然图像形态差异巨大)对模型结构进行针对性调整。
虽然近期有研究开始尝试对模型进行改进,但仍主要局限于传统的卷积神经网络(Convolutional Neural Network, CNN)架构,未能引入近年来在计算机视觉领域取得突破性进展的Transformer架构(如DETR、Deformable DETR等),其潜力有待进一步发掘。

因此,本课题旨在针对上述问题展开深入研究。研究将在技术迁移的基础上,充分考虑更为复杂多变的信号检测场景,从数据处理、模型结构设计到训练策略等环节进行系统性的优化。
其最终目标是实现目标检测技术与信号检测任务的深度融合,显著提升信号检测模型的准确率与鲁棒性。
这一技术在具体实践中将直接赋能两大关键应用:
(1)在DSA中提高信号对总体时频资源的利用率,同时降低信号之间的干扰;
(2)在复杂电磁环境中识别已授权和未授权的信号,并精确定位非法信号,从而保障无线通信的电磁空间安全。

% !TEX root = ../main.tex

% 中英标题:\chapter{中文标题}[英文标题]
\chapter{研究现状及分析}

\section{国内外研究现状}

建议大家仔细阅读示例文档及相关代码,理解插图的方式等。

\lipsum[1]

\section{国内外文献综述及简析}

\subsection{本硕论文题注}[Other picture example]

本硕论文题注如\figref{fig:bm}所示。

\begin{figure}[ht]
\centering
\includegraphics[width = 0.4\textwidth]{golfer}
\caption{打高尔夫球的人,硕士论文要求只用汉语}
\label{fig:bm}
\end{figure}

\subsection{并排图和子图}[Abreast-picture and Sub-picture example]
\subsubsection{并排图}[Abreast-picture example]

使用并排图时,需要注意对齐方式。默认情况是中部对齐。这里给出中部对齐、顶部对齐
、图片底部对齐三种常见方式。其中,底部对齐方式有一个很巧妙的方式,将长度比较小
的图放在左面即可。


\begin{figure}[htbp]
\centering
\begin{minipage}{0.4\textwidth}
\centering
\includegraphics[width=\textwidth]{golfer}
\bicaption[golfer2]{}{打高尔夫球的人}{Fig.$\!$}{The person playing golf}
\end{minipage}
\centering
\begin{minipage}{0.4\textwidth}
\centering
\includegraphics[width=\textwidth]{golfer}
\bicaption[golfer3]{}{打高尔夫球的人。注意,这里默认居中}{Fig.$\!$}{The person playing golf. Please note that, it is vertically center aligned by default.}
\end{minipage}
\end{figure}

\begin{figure}[htbp]
\centering
\begin{minipage}[t]{0.4\textwidth}
\centering
\includegraphics[width=\textwidth]{golfer}
\bicaption[golfer5]{}{打高尔夫球的人}{Fig.$\!$}{The person playing golf}
\end{minipage}
\centering
\begin{minipage}[t]{0.4\textwidth}
\centering
\includegraphics[width=\textwidth]{golfer}
\bicaption[golfer8]{}{打高尔夫球的人。注意,此图是顶部对齐}{Fig.$\!$}{The person playing golf. Please note that, it is vertically top aligned.}
\end{minipage}
\end{figure}

\begin{figure}[htbp]
\centering
\begin{minipage}[t]{0.4\textwidth}
\centering
\includegraphics[width=\textwidth,height=\textwidth]{golfer}
\bicaption[golfer9]{}{打高尔夫球的人。注意,此图对齐方式是图片底部对齐}{Fig.$\!$}{The person playing golf. Please note that, it is vertically bottom aligned for figure.}
\end{minipage}
\centering
\begin{minipage}[t]{0.4\textwidth}
\centering
\includegraphics[width=\textwidth]{golfer}
\bicaption[golfer6]{}{打高尔夫球的人}{Fig.$\!$}{The person playing golf}
\end{minipage}
\end{figure}

\subsubsection{子图}[Sub-picture example]

注意:子图题注也可以只用中文。规范规定“分图题置于分图之下或图题之下”,但没有给出具体的格式要求。
没有要求的另外一个说法就是“无论什么格式都不对”。
所以只有在一个图中有标注“(a),(b)”,无法使用\cs{subfigure}的情况下,使用最后一个图例中的格式设置方法,否则不要使用。
为了应对“无论什么格式都不对”,这个子图图题使用“minipage”和“description”环境,宽度,对齐方式可以按照个人喜好自由设置,是否使用双语子图图题也可以自由设置。

每页底部不要留空白。为此这里添加了无意义文字以演示。Quisque ullamcorper placerat ipsum. Cras nibh. Morbi vel justo vitae lacus tincidunt
ultrices. Lorem ipsum dolor sit amet, consectetuer adipiscing elit. In hac habitasse platea
dictumst. Integer tempus convallis augue. Etiam facilisis. Nunc elementum fermentum
wisi.

\begin{figure}[!ht]
\setlength{\subfigcapskip}{-1bp}
\centering
\begin{minipage}{\textwidth}
\centering
\subfigure{\label{golfer41}}\addtocounter{subfigure}{-2}
\subfigure[The person playing golf]{\subfigure[打高尔夫球的人~1]{\includegraphics[width=0.4\textwidth]{golfer}}}
\hspace{2em}
\subfigure{\label{golfer42}}\addtocounter{subfigure}{-2}
\subfigure[The person playing golf]{\subfigure[打高尔夫球的人~2]{\includegraphics[width=0.4\textwidth]{golfer}}}
\end{minipage}
\centering
\begin{minipage}{\textwidth}
\centering
\subfigure{\label{golfer43}}\addtocounter{subfigure}{-2}
\subfigure[The person playing golf]{\subfigure[打高尔夫球的人~3]{\includegraphics[width=0.4\textwidth]{golfer}}}
\hspace{2em}
\subfigure{\label{golfer44}}\addtocounter{subfigure}{-2}
\subfigure[The person playing golf. Here, 'hang indent' and 'center last line' are not stipulated in the regulation.]{\subfigure[打高尔夫球的人~4。注意,规范中没有明确规定要悬挂缩进、最后一行居中。]{\includegraphics[width=0.4\textwidth]{golfer}}}
\end{minipage}
\vspace{0.2em}
\bicaption[golfer4]{}{打高尔夫球的人}{Fig.$\!$}{The person playing gol}
\end{figure}

\begin{figure}[t]
  \centering
  \begin{minipage}{.7\linewidth}
    \setlength{\subfigcapskip}{-1bp}
    \centering
    \begin{minipage}{\textwidth}
      \centering
      \subfigure{\label{golfer45}}\addtocounter{subfigure}{-2}
      \subfigure[The person playing golf]{\subfigure[打高尔夫球的人~1]{\includegraphics[width=0.4\textwidth]{golfer}}}
      \hspace{4em}
      \subfigure{\label{golfer46}}\addtocounter{subfigure}{-2}
      \subfigure[The person playing golf]{\subfigure[打高尔夫球的人~2]{\includegraphics[width=0.4\textwidth]{golfer}}}
    \end{minipage}
    \vskip 0.2em
  \wuhao 注意:这里是中文图注添加位置(我工要求,图注在图题之上)。
    \vspace{0.2em}
\bicaption[golfer47]{}{打高尔夫球的人。注意,此处我工有另外一处要求,子图图题可以位于主图题之下。但由于没有明确说明位于下方具体是什么格式,所以这里不给出举例。}{Fig.$\!$}{The person playing golf. Please note that, although it is appropriate to put subfigures' captions under this caption as stipulated in regulation, but its format is not clearly stated.}
  \end{minipage}
\end{figure}

\begin{figure}[t]
\centering
% \begin{tikzpicture}
% 	\node[anchor=south west,inner sep=0] (image) at (0,0) {\includegraphics[width=0.3\textwidth]{golfer}};
% 	\begin{scope}[x={(image.south east)},y={(image.north west)}]
% 		\node at (0.3,0.5) {a)};
% 		\node at (0.8,0.2) {b)};
% 	\end{scope}
% \end{tikzpicture}
\includegraphics[width=0.3\textwidth]{golfer}
\bicaption[golfer0]{}{打高尔夫球球的人(博士论文双语题注)}{Fig.$\!$}{The person playing golf (Doctoral thesis)}
\vskip -0.4em
 \hspace{2em}
\begin{minipage}[t]{0.3\textwidth}
\wuhao \setlist[description]{font=\normalfont}
	\begin{description}
		\item[(a)]子图图题
		\item[(a)]Subfigure caption
	\end{description}
 \end{minipage}
 \hspace{2em}
 \begin{minipage}[t]{0.3\textwidth}
\wuhao \setlist[description]{font=\normalfont}
	\begin{description}
		\item[(b)]子图图题
		\item[(b)]Subfigure caption
	\end{description}
\end{minipage}
\end{figure}

\begin{figure}[!ht]
	\centering
	\begin{sideways}
		\begin{minipage}{\textheight}
			\centering
			\fbox{\includegraphics[width=0.2\textwidth]{golfer}}
			\fbox{\includegraphics[width=0.2\textwidth]{golfer}}
			\fbox{\includegraphics[width=0.2\textwidth]{golfer}}
			\fbox{\includegraphics[width=0.2\textwidth]{golfer}}
			\fbox{\includegraphics[width=0.2\textwidth]{golfer}}
			\fbox{\includegraphics[width=0.2\textwidth]{golfer}}
			\fbox{\includegraphics[width=0.2\textwidth]{golfer}}
\bicaption[golfer7]{}{打高尔夫球的人(非规范要求)}{Fig.$\!$}{The person playing golf (Not stated in the regulation)}
		\end{minipage}
	\end{sideways}
\end{figure}

\clearpage

如果不想让图片浮动到下一章节,那么在此处使用\cs{clearpage}命令。

\subsection{如何做出符合规范的漂亮的图}

关于作图工具在后文\ref{drawtool}中给出一些作图工具的介绍,此处不多言。
此处以R语言和Tikz为例说明如何做出符合规范的图。

\subsubsection{Tikz作图}

使用Tikz作图核心思想是把格式、主题、样式与内容分离,定义在全局中。
注意字体设置可以有两种选择,如果字少,用五号字,字多用小五。
使用Tikz作图不会出现字体问题,字体会自动与正文一致。

\subsubsection{R作图}

R是一种极具有代表性的典型的作图工具,应用广泛。
与Tikz图不同,R作图分两种情况:(1)可以转换为Tikz码;(2)不可转换为Tikz码。
第一种情况图形简单,图形中不含有很多数据点,使用R语言中的Tikz包即可。
第二种情况是图形复杂,含有海量数据点,这时候不要转成Tikz矢量图,这会使得论文体积巨大。
推荐使用pdf或png非矢量图形。
使用非矢量图形时要注意选择好字号(五号或小五),和字体(宋体、新罗马)然后选择生成图形大小,注意此时在正文中使用\cs{includegraphics}命令导入时,不要像导入矢量图那样控制图形大小,使用图形的原本的
宽度和高度,这样就确保了非矢量图形中的文字与正文一致了。

为了控制\hitszthesis\ 的大小,此处不给出具体举例。

\subsubsection{专业绘图工具}
\label{drawtool}

推荐使用tikz包,使用tikz源码绘图的好处是,图片中的字体与正文中的字体一致。具体如
何使用tikz绘图不属于模板范畴。

tikz适合用来画不需要大量实验数据支撑示意图。但R语言等专业绘图工具具有画出各种、
专业、复杂的数据图。R语言中有tikz包,能自动生成tikz码,这样tikz几乎无所不能。
对于排版有极致追求的小伙伴,可以参考
\href{http://www.texample.net/tikz/resources/}{http://www.texample.net/tikz/resources/}
中所列工具,几乎所有作图软件所作的图形都可转成tikz,然后可以自由的在tikz中修改
图中内容,定义字体等等。实现前文窝工规范中要求的图中字体的一致性的终极目标。


% !TEX root = ../main.tex

% 中英标题:\chapter{中文标题}[英文标题]
\chapter{主要研究内容及研究方案}

% ========== 3.1 研究内容 ==========
\section{研究内容}

\subsection{针对不同尺度信号数据的处理方法研究}
在在真实的电磁监测场景中,采样率、采样时长等采集参数的多样性,对数据处理提出了关键挑战。
为保证神经网络能够有效学习信号的内在模式,进行STFT时,必须设定固定的频率分辨率与时间分辨率。
然而,这一约束将不可避免地导致采用不同采样参数的样本在STFT后生成尺寸大小不一的二维时频谱图。

传统的计算机视觉模型,特别是包含全连接层的网络,通常要求输入图像具有固定的尺寸。
尽管可通过插值缩放或统一尺寸填充(Padding)来强制对齐,但这两种方法均不适用于本任务:
前者会严重扭曲信号固有的时频结构,破坏其物理意义;
后者在样本尺寸差异悬殊时,会引入大量无信息的零值填充,不仅造成计算资源的极大浪费,还可能干扰模型的有效特征学习。

针对此问题,本研究拟对目标检测领域的多尺度训练(Multiscale Training)策略进行创新性改造。
不同于在视觉任务中将其作为一种可选的数据增强手段,本课题将多尺度处理升格为一种应对异构数据源的必要核心机制。
具体而言,本研究将充分利用所选DETR模型(由CNN骨干网络、Transformer及预测层构成)对输入尺寸无严格限制的结构优势,
在训练流程中设计一种自适应批处理(Adaptive Batching)策略,即将尺寸相同的STFT数据动态地组织成批次(Batch)送入网络。
该策略旨在使模型能够稳健、高效地处理来源多样化的真实信号数据,从根本上解决异构尺度输入的难题。

\subsection{基于Transformer检测头的信号检测模型研究}
检测头(Detection Head),即负责从深度特征图中解码出目标位置和类别的网络末端部分,是决定检测器性能的关键组件。
主流的检测头可依据其结构特性分为三种范式,且各有优劣:
(1)全连接检测头:其优点在于可以人为设定固定数量的预测输出,无需复杂的后处理。但其结构要求输入特征图必须为固定尺寸,这与前述的多尺度输入策略相悖,缺乏灵活性。
(2)全卷积检测头:该结构能够灵活地适应不同尺寸的输入,但其输出的预测框数量不固定,与输入尺寸相关,因此必须依赖NMS等复杂的后处理算法来滤除冗余的预测框。
(3)Transformer检测头:该结构结合了前两者的优点:既能处理可变尺寸的输入,又能通过其内部的对象查询(Object Queries)机制输出固定数量的预测结果,从而省去了复杂的后处理环节,构建了真正意义上的端到端检测流程。

更重要的是,Transformer架构的核心——自注意力机制(Self-Attention),赋予了模型强大的全局依赖建模能力,这对于捕捉信号间的谐波关系、跳频模式等全局性特征具有潜在优势。
然而,从时频谱图的视觉先验来看,信号的形态多为局部性较强的矩形或条状。
全局注意力机制是否会引入不必要的计算冗余,以及其对于信号局部特征的捕捉是否优于卷积网络,是一个需要通过实验深入探究的关键问题。
因此,本研究将重点构建并评估基于Transformer的检测头,以验证其在信号检测任务中的实际效能与适用性。

\subsection{针对窄带短时信号的模型优化策略研究}
在时频谱图中,窄带短时信号(如猝发信号、短时通信等)表现为尺寸微小、持续时间短的“小目标”,这对检测模型提出了极高的要求。
标准的检测模型由于特征图分辨率较低、感受野过大等问题,在处理这类小目标时常出现漏检的现象。
为攻克这一难题,本研究拟引入Deformable DETR模型进行优化:
(1)Deformable DETR借鉴了FPN的思想,融合了骨干网络在不同阶段输出的多尺度特征图。
高层特征图包含丰富的语义信息,有助于信号分类,底层特征图具有更高的空间分辨率,保留了精确的位置信息。
通过将二者有效结合,模型能够同时兼顾对大、小不同尺度信号的检测能力;
(2)标准的Transformer注意力机制会对特征图上的所有像素点进行计算,对于仅占图像一小部分的稀疏信号而言,这会带来巨大的计算浪费。
Deformable DETR则是采用了可变形注意力机制,即不再对整个特征图进行全局计算,而是让网络根据输入动态地学习少数关键采样点的位置,并将注意力集中在这些与目标最相关的区域。
这不仅大幅降低了模型的计算复杂度和内存消耗,还可以使得注意力能够更精准地聚焦于目标区域,从而显著提升对窄带短时信号这一类小目标的检测精度。

\section{研究方案}

为实现上述研究目标,本课题计划遵循以下技术路线,分阶段进行:

% \begin{figure}[htbp]
% \centering
% \begin{minipage}{0.4\textwidth}
% \centering
% \includegraphics[width=\textwidth]{technical_route}
% \bicaption[golfer2]{}{打高尔夫球的人}{Fig.$\!$}{The person playing golf}
% \end{minipage}
% \centering
% \end{figure}

\begin{figure}[htbp]
\centering
\includegraphics[width=\textwidth]{technology roadmap}
\bicaption[{technology roadmap}]{}{技术路线图}{Fig.$\!$}{technology roadmap}
\end{figure}

% \begin{figure}[htbp]
% \centering
% \makebox[\textwidth][c]{%
%     \includegraphics[width=1.2\textwidth]{technical_route}%
% }
% \bicaption[golfer2]{}{打高尔夫球的人}{Fig.$\!$}{The person playing golf}
% \end{figure}


1. 数据集分析与预处理:
(1)对现有的真实电磁环境信号数据集进行探索性数据分析(Exploratory Data Analysis, EDA),明确其中信号制式类型的均衡性、不同信号的时宽与带宽分布、以及单个样本中的信号密度等关键特征。
(2)通过STFT将一维时域采样数据转换为二维时频谱图。在此过程中,需要仔细选择窗函数、窗长及重叠率等参数,以寻求时间和频率分辨率之间的最佳平衡。
(3)对部分样本进行可视化观察,以直观检验STFT参数的合理性,并确认是否需要对生成的时频谱图进行幅度归一化(如Min-Max归一化或Z-Score归一化)处理,以利于模型的稳定训练与快速收敛。

2. 基线模型搭建与实验:
为了科学、定量地评估本课题所提方法的有效性,将首先搭建一个基线(Baseline)模型。
考虑到YOLO系列在目标检测领域中速度与精度的良好平衡性,拟选择YOLOv11作为基线模型。
将在预处理完成的数据集上对其进行训练与测试,并详细记录其平均精度均值(mAP)、精确率(Precision)、召回率(Recall)以及检测速度(FPS)等核心性能指标,
为后续所有改进模型提供用于比较和评估的参照标准。

3. 核心模型的设计与实现:
本阶段是研究的核心,将重点围绕Deformable DETR模型展开,将其迁移至信号时频检测这一特定场景并进行适配与优化。
(1)搭建基于Deformable DETR的信号检测框架,该框架由CNN骨干网络、集成了多尺度特征融合与可变形注意力的Transformer编解码器,以及用于目标分类和边界框回归的线性预测层构成。
(2)针对性地解决多尺度输入问题,在训练流程中设计数据加载策略,将尺寸相同的时频谱图组织在同一Batch中送入网络,充分利用模型对可变尺寸输入的兼容性。
(3)将根据信号数据的特性对模型进行微调,例如调整Transformer中的对象查询数量以匹配数据集中信号的典型密度,或优化损失函数中分类损失与回归损失的权重,使模型更专注于本任务的特定挑战。

4. 综合实验与性能评估:
在完成核心模型的设计与实现后,通过多维度实验,全面、深入地验证核心模型的性能与有效性。
(1)整体性能对比:将所提出的Deformable DETR模型与基线模型(YOLOv11)在同一测试集上进行比较,分析两者在mAP、精确率、召回率等关键指标上的差异,验证核心模型的总体优越性。
(2)消融实验: 为验证模型关键组件的有效性,将设计消融研究。例如,通过对比Deformable DETR与使用单尺度特征的基础版DETR的性能,来量化多尺度特征融合(FPN思想)与可变形注意力机制对于检测窄带短时“小目标”信号的具体提升效果。
(3)定性分析: 除了定量的指标对比,还将对检测结果进行可视化分析。重点选取一些具有挑战性的样本,如包含低信噪比信号、密集重叠信号或极端长宽比信号的频谱图,直观对比不同模型在这些复杂场景下的检测效果,以展示本研究模型在鲁棒性与精确性上的优势。

5. 总结结论与论文撰写:
系统性地整理和总结所有实验数据与分析结果,凝练出本研究的核心结论,
明确所提出的信号检测模型相较于传统方法和基线模型的优势所在,并客观分析其可能存在的局限性。
在此基础上,将对未来可能的研究方向进行展望,并依据学位论文的规范要求完成毕业设计的撰写工作。


% !TEX root = ../main.tex

\chapter{进度安排及预期目标}

% ========== 4.1 进度安排 ==========
\section{进度安排}
本课题的研究时间为2025年9月开始至2026年5月,根据之前的项目经验,结合自身的学习科研能力,针对本课题研究目标,特制定以下研究进度计划:

1. 2025年9月至2025年10月:
(1)深入研读相关领域的前沿文献,完善并最终确定详细的技术实现方案;
(2)对信号数据集进行全面的统计性分析,完成STFT转换、数据清洗、归一化及标注格式统一等所有预处理工作;
(3)完成深度学习实验环境的搭建、配置与调试,并验证数据加载与基础训练流程的稳定性。

2. 2025年11月至2026年3月:
(1)实现YOLOv11基线模型与基于Deformable DETR的核心检测模型;
(2)解决多尺度时频谱图的输入与批处理问题,
(3)在数据集上完成模型的完整训练、调优与多维度性能评估,获取全面的实验数据,并进行系统性的对比分析,得出研究结论。

3. 2026年4月至2026年5月:
(1)根据已有的实验结果,撰写毕业论文,准备毕业设计答辩。

% ========== 4.2 预期目标 ==========
\section{预期目标}
本课题旨在通过上述研究,达成以下具体目标:

1. 深入分析真实电磁环境信号数据集的关键特性,完成科学的STFT数据预处理。
最终构建一套能够支撑异构尺度输入训练的、规范化的实验与验证数据集,为后续研究提供坚实的数据基础。

2. 成功复现并将在信号时频谱图检测任务上适配、调优一个高性能的YOLOv11模型。
获取其在各项关键指标(精确率、召回率、mAP等)上的可靠性能数据,为衡量本课题核心模型的先进性提供一个强有力的参照基准。

3. 成功将先进的端到端目标检测框架Deformable DETR迁移并优化,使其深度适配信号时频联合检测任务。
通过充分的实验对比,验证该模型相较于传统CNN基线模型,在复杂电磁环境下,特别是在处理密集、微小及重叠信号时的鲁棒性与性能优越性。

4. 针对真实场景中STFT时频谱图尺寸不一的核心工程挑战,成功设计并实现一种创新的多尺度自适应训练策略。
该策略旨在从根本上解决异构数据输入的难题,避免因传统处理方式(插值、填充)导致的特征失真与资源浪费,提升模型的工程实用性。
% !TEX root = ../main.tex

\chapter{已具备和所需的条件和经费}

% ========== 5.1 已具备的条件 ==========
\section{已具备的条件}

\subsection{实验室已具备条件}
本研究依托实验室现有的高性能计算服务器。

该服务器配备了128GB系统内存与NVIDIA RTX 4090高性能图形处理器(GPU),其强大的计算能力与充足的显存资源,能够为本课题中大规模数据集的处理及深度学习模型的训练提供充分的硬件支持。
且服务器已部署了稳定、完善的科研环境,包括Ubuntu操作系统、Python编程环境以及PyTorch等主流深度学习框架。相关开发库与依赖项均已配置完毕,可确保研究工作的顺利开展。

服务器已部署了稳定、完善的科研环境,包括Ubuntu操作系统、Python编程环境以及PyTorch等主流深度学习框架。相关开发库与依赖项均已配置完毕,可确保研究工作的顺利开展。

鉴于本课题的核心工作为算法设计与软件实现,且所需硬件与软件条件均已具备,因此本研究无需申请额外经费支持,现有资源可完全保障课题的顺利完成。

\subsection{实验室经费保障}
本课题主要通过编写代码完成,同时训练所需的硬件设备均以具备,故无需经费支持。

% ========== 5.2 所需的条件 ==========
\section{所需的条件}
所需条件已经在5.1中详细列出并已具备,无需另外条件和经费。
% !TEX root = ../main.tex

\chapter{预计困难及解决方案说明}

% ========== 6.1 技术难点与预计困难 ==========
\section{技术难点与预计困难}

1. 数据的异构性与复杂性:
本研究所采用的数据集包含由不同采样参数导致的异构尺寸时频谱图,如何设计高效的数据加载与批处理机制以适应这种内在的异构性,是一个关键的工程挑战。
此外,数据集中信号类型多样、时频密集且存在相互重叠,这对模型的特征表征与精细分辨能力提出了极高的要求。

2. 模型的高复杂性与超参数调优:
本研究中超参数调优的复杂度较高,涉及模型结构参数(如模块层数)、优化器与学习率调度器参数(如学习率、Warmup轮数)、以及损失函数各部分权重系数等众多变量。
不恰当的超参数配置极易导致模型不收敛或陷入局部最优,因此,高效的超参数寻优策略至关重要。

3. 微小信号检测的瓶颈:
在时频谱图中,大量关键信号(如猝发信号)呈现为时宽与带宽极窄的“小目标”。
此类目标在经过骨干网络的多层下采样后,其特征信息极易被稀释或丢失,是目标检测领域的公认技术瓶颈。
如何有效克服这一瓶颈,确保模型对微小信号的检出率与定位精度,是衡量本研究算法先进性的核心指标。

4. 训练开销与显存瓶颈:
引入Transformer架构将不可避免地带来计算量与显存占用的显著增长。
在有限的硬件资源下,为防止显存溢出而过度减小批处理大小(Batch Size),不仅会严重降低训练效率,还可能损害模型的收敛性能与最终精度。

% ========== 6.2 解决方案 ==========
\section{解决方案}

针对上述可能遇到的困难,本课题预先制定了如下应对策略和解决方案:

1. 将设计并实现一个定制化的数据整理函数(Collate Function)。
该函数能够在构建每个训练批次时,动态地将尺寸相同或相近的样本聚合,从而在不破坏原始数据结构的前提下,实现对异构数据的兼容与高效处理。

2. 将采用迁移学习范式,加载在大型公开数据集上预训练的模型权重作为优化起点,以加速收敛并提升性能。
同时,将借鉴相关领域顶级会议论文中的成熟训练策略与超参数配置,构建一个可靠的初始设定。
采用控制变量法进行系统性的超参数调优,并密切监控训练过程中的损失曲线与验证集性能。
辅以早停(Early Stopping)策略避免不必要的计算开销,并通过对模型输出结果的可视化诊断,为针对性的结构与损失函数调整提供直观依据。

3. 充分利用Deformable DETR的多尺度特征融合能力,确保来自底层网络的高分辨率特征被有效传递至检测头。
发挥可变形注意力机制的优势,引导模型将计算资源精准聚焦于微小信号所在的关键区域。

4. 将优先采用混合精度训练(Automatic Mixed Precision, AMP),在几乎不损失模型精度的前提下,大幅降低显存占用并提升训练速度。
若批大小依然受限,将采用梯度累积技术,以时间换空间,在不增加显存消耗的情况下实现等效的大批次训练效果。
在必要时,可利用实验室的多GPU资源,通过数据并行(Data Parallelism)等分布式训练框架,从根本上扩展可用计算与显存资源,突破单卡硬件瓶颈。

% 以下为中期的章节内容

% % !TEX root = ../main.tex

\chapter{课题背景及研究的目的与意义}

% ========== 1.1 课题背景 ==========
\section{课题背景}

随着物联网(Internet of Things, IoT)技术的迅猛发展,无线通信设备数量正呈爆炸式增长。
从智能家居、工业自动化到智慧医疗与无人驾驶,海量的无线设备与通信技术在有限的频谱资源中交织共存,构成了日益拥挤且动态多变的复杂电磁环境。
在此背景下,信号检测(Signal Detection)技术的重要性日益凸显,它不仅是提升频谱共享效率、降低信号间干扰的关键,也是保障无线通信安全、监控非法信号的必要手段。

与仅关注频段是否被占用的传统频谱感知(Spectrum Sensing, SS)不同,
信号检测旨在同时完成两项核心子任务:
(1)对环境中的信号进行精确的时频定位(即确定信号在何时占用哪些频谱资源);
(2)对定位到的信号进行准确的制式识别(即判断信号的调制类型或通信协议)。
由此可知,信号检测是实现动态频谱接入(Dynamic Spectrum Access, DSA)和保障通信可靠性的关键前提。

然而,能量检测、匹配滤波的传统频段定位方式和基于机器学习的传统信号分类方式由于其对噪声功率敏感、依赖信号先验信息以及计算复杂度高等固有局限,
难以完全适应当前非协作、高动态的复杂电磁环境。

近年来,深度学习(Deep Learning, DL)领域的突破性进展为应对上述挑战提供了全新思路。
通过将一维时域信号经过短时傅里叶变换(Short-Time Fourier Transform, STFT)转化为二维时频图,
信号感知任务得以巧妙地转化为计算机视觉领域中技术成熟的目标检测(Object Detection)问题。
这种跨域迁移方法的优势在于,
它能够借助目标检测模型实现端到端的处理,在无需精确信号或信道先验知识的条件下,自动学习信号在时频域的深层特征,从而在信号的精确定位与制式分类方面展现出巨大潜力。

% ========== 1.2 研究的目的与意义 ==========
\section{研究的目的与意义}

尽管计算机视觉领域的目标检测技术已日趋成熟,并涌现出众多先进方案,
但其在信号检测领域的应用仍处于初步探索阶段,不仅相关研究成果有限,且现有的技术迁移模式也存在若干亟待解决的问题:

1. 数据集与真实场景存在偏差:
目前,该领域的研究大多采用仿真生成或混合真实信号的数据集。
这些数据集普遍存在对现实场景的过度简化,例如信号制式与调制类型有限、信号时频密度较低,且通常忽略了信号间的重叠与干扰。
此外,现有数据集普遍假设了固定的采样率与采样时长,这与实际监测中参数多变的情况存在显著差异。
虽然这种简化有助于在研究初期快速验证算法的可行性,但却使其难以准确评估模型在真实复杂电磁环境下的性能与泛化能力。

2. 模型迁移缺乏针对性优化:
现有的研究在模型迁移方面,大多直接套用计算机视觉领域的经典模型(如Faster R-CNN),而未根据信号时频图的内在特征(相较于自然图像形态差异巨大)对模型结构进行针对性调整。
虽然近期有研究开始尝试对模型进行改进,但仍主要局限于传统的卷积神经网络(Convolutional Neural Network, CNN)架构,未能引入近年来在计算机视觉领域取得突破性进展的Transformer架构(如DETR、Deformable DETR等),其潜力有待进一步发掘。

因此,本课题旨在针对上述问题展开深入研究。研究将在技术迁移的基础上,充分考虑更为复杂多变的信号检测场景,从数据处理、模型结构设计到训练策略等环节进行系统性的优化。
其最终目标是实现目标检测技术与信号检测任务的深度融合,显著提升信号检测模型的准确率与鲁棒性。
这一技术在具体实践中将直接赋能两大关键应用:
(1)在DSA中提高信号对总体时频资源的利用率,同时降低信号之间的干扰;
(2)在复杂电磁环境中识别已授权和未授权的信号,并精确定位非法信号,从而保障无线通信的电磁空间安全。


% % !TEX root = ../main.tex

% 中英标题:\chapter{中文标题}[英文标题]
\chapter{研究现状及分析}

\section{国内外研究现状}

建议大家仔细阅读示例文档及相关代码,理解插图的方式等。

\lipsum[1]

\section{国内外文献综述及简析}

\subsection{本硕论文题注}[Other picture example]

本硕论文题注如\figref{fig:bm}所示。

\begin{figure}[ht]
\centering
\includegraphics[width = 0.4\textwidth]{golfer}
\caption{打高尔夫球的人,硕士论文要求只用汉语}
\label{fig:bm}
\end{figure}

\subsection{并排图和子图}[Abreast-picture and Sub-picture example]
\subsubsection{并排图}[Abreast-picture example]

使用并排图时,需要注意对齐方式。默认情况是中部对齐。这里给出中部对齐、顶部对齐
、图片底部对齐三种常见方式。其中,底部对齐方式有一个很巧妙的方式,将长度比较小
的图放在左面即可。


\begin{figure}[htbp]
\centering
\begin{minipage}{0.4\textwidth}
\centering
\includegraphics[width=\textwidth]{golfer}
\bicaption[golfer2]{}{打高尔夫球的人}{Fig.$\!$}{The person playing golf}
\end{minipage}
\centering
\begin{minipage}{0.4\textwidth}
\centering
\includegraphics[width=\textwidth]{golfer}
\bicaption[golfer3]{}{打高尔夫球的人。注意,这里默认居中}{Fig.$\!$}{The person playing golf. Please note that, it is vertically center aligned by default.}
\end{minipage}
\end{figure}

\begin{figure}[htbp]
\centering
\begin{minipage}[t]{0.4\textwidth}
\centering
\includegraphics[width=\textwidth]{golfer}
\bicaption[golfer5]{}{打高尔夫球的人}{Fig.$\!$}{The person playing golf}
\end{minipage}
\centering
\begin{minipage}[t]{0.4\textwidth}
\centering
\includegraphics[width=\textwidth]{golfer}
\bicaption[golfer8]{}{打高尔夫球的人。注意,此图是顶部对齐}{Fig.$\!$}{The person playing golf. Please note that, it is vertically top aligned.}
\end{minipage}
\end{figure}

\begin{figure}[htbp]
\centering
\begin{minipage}[t]{0.4\textwidth}
\centering
\includegraphics[width=\textwidth,height=\textwidth]{golfer}
\bicaption[golfer9]{}{打高尔夫球的人。注意,此图对齐方式是图片底部对齐}{Fig.$\!$}{The person playing golf. Please note that, it is vertically bottom aligned for figure.}
\end{minipage}
\centering
\begin{minipage}[t]{0.4\textwidth}
\centering
\includegraphics[width=\textwidth]{golfer}
\bicaption[golfer6]{}{打高尔夫球的人}{Fig.$\!$}{The person playing golf}
\end{minipage}
\end{figure}

\subsubsection{子图}[Sub-picture example]

注意:子图题注也可以只用中文。规范规定“分图题置于分图之下或图题之下”,但没有给出具体的格式要求。
没有要求的另外一个说法就是“无论什么格式都不对”。
所以只有在一个图中有标注“(a),(b)”,无法使用\cs{subfigure}的情况下,使用最后一个图例中的格式设置方法,否则不要使用。
为了应对“无论什么格式都不对”,这个子图图题使用“minipage”和“description”环境,宽度,对齐方式可以按照个人喜好自由设置,是否使用双语子图图题也可以自由设置。

每页底部不要留空白。为此这里添加了无意义文字以演示。Quisque ullamcorper placerat ipsum. Cras nibh. Morbi vel justo vitae lacus tincidunt
ultrices. Lorem ipsum dolor sit amet, consectetuer adipiscing elit. In hac habitasse platea
dictumst. Integer tempus convallis augue. Etiam facilisis. Nunc elementum fermentum
wisi.

\begin{figure}[!ht]
\setlength{\subfigcapskip}{-1bp}
\centering
\begin{minipage}{\textwidth}
\centering
\subfigure{\label{golfer41}}\addtocounter{subfigure}{-2}
\subfigure[The person playing golf]{\subfigure[打高尔夫球的人~1]{\includegraphics[width=0.4\textwidth]{golfer}}}
\hspace{2em}
\subfigure{\label{golfer42}}\addtocounter{subfigure}{-2}
\subfigure[The person playing golf]{\subfigure[打高尔夫球的人~2]{\includegraphics[width=0.4\textwidth]{golfer}}}
\end{minipage}
\centering
\begin{minipage}{\textwidth}
\centering
\subfigure{\label{golfer43}}\addtocounter{subfigure}{-2}
\subfigure[The person playing golf]{\subfigure[打高尔夫球的人~3]{\includegraphics[width=0.4\textwidth]{golfer}}}
\hspace{2em}
\subfigure{\label{golfer44}}\addtocounter{subfigure}{-2}
\subfigure[The person playing golf. Here, 'hang indent' and 'center last line' are not stipulated in the regulation.]{\subfigure[打高尔夫球的人~4。注意,规范中没有明确规定要悬挂缩进、最后一行居中。]{\includegraphics[width=0.4\textwidth]{golfer}}}
\end{minipage}
\vspace{0.2em}
\bicaption[golfer4]{}{打高尔夫球的人}{Fig.$\!$}{The person playing gol}
\end{figure}

\begin{figure}[t]
  \centering
  \begin{minipage}{.7\linewidth}
    \setlength{\subfigcapskip}{-1bp}
    \centering
    \begin{minipage}{\textwidth}
      \centering
      \subfigure{\label{golfer45}}\addtocounter{subfigure}{-2}
      \subfigure[The person playing golf]{\subfigure[打高尔夫球的人~1]{\includegraphics[width=0.4\textwidth]{golfer}}}
      \hspace{4em}
      \subfigure{\label{golfer46}}\addtocounter{subfigure}{-2}
      \subfigure[The person playing golf]{\subfigure[打高尔夫球的人~2]{\includegraphics[width=0.4\textwidth]{golfer}}}
    \end{minipage}
    \vskip 0.2em
  \wuhao 注意:这里是中文图注添加位置(我工要求,图注在图题之上)。
    \vspace{0.2em}
\bicaption[golfer47]{}{打高尔夫球的人。注意,此处我工有另外一处要求,子图图题可以位于主图题之下。但由于没有明确说明位于下方具体是什么格式,所以这里不给出举例。}{Fig.$\!$}{The person playing golf. Please note that, although it is appropriate to put subfigures' captions under this caption as stipulated in regulation, but its format is not clearly stated.}
  \end{minipage}
\end{figure}

\begin{figure}[t]
\centering
% \begin{tikzpicture}
% 	\node[anchor=south west,inner sep=0] (image) at (0,0) {\includegraphics[width=0.3\textwidth]{golfer}};
% 	\begin{scope}[x={(image.south east)},y={(image.north west)}]
% 		\node at (0.3,0.5) {a)};
% 		\node at (0.8,0.2) {b)};
% 	\end{scope}
% \end{tikzpicture}
\includegraphics[width=0.3\textwidth]{golfer}
\bicaption[golfer0]{}{打高尔夫球球的人(博士论文双语题注)}{Fig.$\!$}{The person playing golf (Doctoral thesis)}
\vskip -0.4em
 \hspace{2em}
\begin{minipage}[t]{0.3\textwidth}
\wuhao \setlist[description]{font=\normalfont}
	\begin{description}
		\item[(a)]子图图题
		\item[(a)]Subfigure caption
	\end{description}
 \end{minipage}
 \hspace{2em}
 \begin{minipage}[t]{0.3\textwidth}
\wuhao \setlist[description]{font=\normalfont}
	\begin{description}
		\item[(b)]子图图题
		\item[(b)]Subfigure caption
	\end{description}
\end{minipage}
\end{figure}

\begin{figure}[!ht]
	\centering
	\begin{sideways}
		\begin{minipage}{\textheight}
			\centering
			\fbox{\includegraphics[width=0.2\textwidth]{golfer}}
			\fbox{\includegraphics[width=0.2\textwidth]{golfer}}
			\fbox{\includegraphics[width=0.2\textwidth]{golfer}}
			\fbox{\includegraphics[width=0.2\textwidth]{golfer}}
			\fbox{\includegraphics[width=0.2\textwidth]{golfer}}
			\fbox{\includegraphics[width=0.2\textwidth]{golfer}}
			\fbox{\includegraphics[width=0.2\textwidth]{golfer}}
\bicaption[golfer7]{}{打高尔夫球的人(非规范要求)}{Fig.$\!$}{The person playing golf (Not stated in the regulation)}
		\end{minipage}
	\end{sideways}
\end{figure}

\clearpage

如果不想让图片浮动到下一章节,那么在此处使用\cs{clearpage}命令。

\subsection{如何做出符合规范的漂亮的图}

关于作图工具在后文\ref{drawtool}中给出一些作图工具的介绍,此处不多言。
此处以R语言和Tikz为例说明如何做出符合规范的图。

\subsubsection{Tikz作图}

使用Tikz作图核心思想是把格式、主题、样式与内容分离,定义在全局中。
注意字体设置可以有两种选择,如果字少,用五号字,字多用小五。
使用Tikz作图不会出现字体问题,字体会自动与正文一致。

\subsubsection{R作图}

R是一种极具有代表性的典型的作图工具,应用广泛。
与Tikz图不同,R作图分两种情况:(1)可以转换为Tikz码;(2)不可转换为Tikz码。
第一种情况图形简单,图形中不含有很多数据点,使用R语言中的Tikz包即可。
第二种情况是图形复杂,含有海量数据点,这时候不要转成Tikz矢量图,这会使得论文体积巨大。
推荐使用pdf或png非矢量图形。
使用非矢量图形时要注意选择好字号(五号或小五),和字体(宋体、新罗马)然后选择生成图形大小,注意此时在正文中使用\cs{includegraphics}命令导入时,不要像导入矢量图那样控制图形大小,使用图形的原本的
宽度和高度,这样就确保了非矢量图形中的文字与正文一致了。

为了控制\hitszthesis\ 的大小,此处不给出具体举例。

\subsubsection{专业绘图工具}
\label{drawtool}

推荐使用tikz包,使用tikz源码绘图的好处是,图片中的字体与正文中的字体一致。具体如
何使用tikz绘图不属于模板范畴。

tikz适合用来画不需要大量实验数据支撑示意图。但R语言等专业绘图工具具有画出各种、
专业、复杂的数据图。R语言中有tikz包,能自动生成tikz码,这样tikz几乎无所不能。
对于排版有极致追求的小伙伴,可以参考
\href{http://www.texample.net/tikz/resources/}{http://www.texample.net/tikz/resources/}
中所列工具,几乎所有作图软件所作的图形都可转成tikz,然后可以自由的在tikz中修改
图中内容,定义字体等等。实现前文窝工规范中要求的图中字体的一致性的终极目标。



% % !TEX root = ../main.tex

% 中英标题:\chapter{中文标题}[英文标题]
\chapter{主要研究内容及研究方案}

% ========== 3.1 研究内容 ==========
\section{研究内容}

\subsection{针对不同尺度信号数据的处理方法研究}
在在真实的电磁监测场景中,采样率、采样时长等采集参数的多样性,对数据处理提出了关键挑战。
为保证神经网络能够有效学习信号的内在模式,进行STFT时,必须设定固定的频率分辨率与时间分辨率。
然而,这一约束将不可避免地导致采用不同采样参数的样本在STFT后生成尺寸大小不一的二维时频谱图。

传统的计算机视觉模型,特别是包含全连接层的网络,通常要求输入图像具有固定的尺寸。
尽管可通过插值缩放或统一尺寸填充(Padding)来强制对齐,但这两种方法均不适用于本任务:
前者会严重扭曲信号固有的时频结构,破坏其物理意义;
后者在样本尺寸差异悬殊时,会引入大量无信息的零值填充,不仅造成计算资源的极大浪费,还可能干扰模型的有效特征学习。

针对此问题,本研究拟对目标检测领域的多尺度训练(Multiscale Training)策略进行创新性改造。
不同于在视觉任务中将其作为一种可选的数据增强手段,本课题将多尺度处理升格为一种应对异构数据源的必要核心机制。
具体而言,本研究将充分利用所选DETR模型(由CNN骨干网络、Transformer及预测层构成)对输入尺寸无严格限制的结构优势,
在训练流程中设计一种自适应批处理(Adaptive Batching)策略,即将尺寸相同的STFT数据动态地组织成批次(Batch)送入网络。
该策略旨在使模型能够稳健、高效地处理来源多样化的真实信号数据,从根本上解决异构尺度输入的难题。

\subsection{基于Transformer检测头的信号检测模型研究}
检测头(Detection Head),即负责从深度特征图中解码出目标位置和类别的网络末端部分,是决定检测器性能的关键组件。
主流的检测头可依据其结构特性分为三种范式,且各有优劣:
(1)全连接检测头:其优点在于可以人为设定固定数量的预测输出,无需复杂的后处理。但其结构要求输入特征图必须为固定尺寸,这与前述的多尺度输入策略相悖,缺乏灵活性。
(2)全卷积检测头:该结构能够灵活地适应不同尺寸的输入,但其输出的预测框数量不固定,与输入尺寸相关,因此必须依赖NMS等复杂的后处理算法来滤除冗余的预测框。
(3)Transformer检测头:该结构结合了前两者的优点:既能处理可变尺寸的输入,又能通过其内部的对象查询(Object Queries)机制输出固定数量的预测结果,从而省去了复杂的后处理环节,构建了真正意义上的端到端检测流程。

更重要的是,Transformer架构的核心——自注意力机制(Self-Attention),赋予了模型强大的全局依赖建模能力,这对于捕捉信号间的谐波关系、跳频模式等全局性特征具有潜在优势。
然而,从时频谱图的视觉先验来看,信号的形态多为局部性较强的矩形或条状。
全局注意力机制是否会引入不必要的计算冗余,以及其对于信号局部特征的捕捉是否优于卷积网络,是一个需要通过实验深入探究的关键问题。
因此,本研究将重点构建并评估基于Transformer的检测头,以验证其在信号检测任务中的实际效能与适用性。

\subsection{针对窄带短时信号的模型优化策略研究}
在时频谱图中,窄带短时信号(如猝发信号、短时通信等)表现为尺寸微小、持续时间短的“小目标”,这对检测模型提出了极高的要求。
标准的检测模型由于特征图分辨率较低、感受野过大等问题,在处理这类小目标时常出现漏检的现象。
为攻克这一难题,本研究拟引入Deformable DETR模型进行优化:
(1)Deformable DETR借鉴了FPN的思想,融合了骨干网络在不同阶段输出的多尺度特征图。
高层特征图包含丰富的语义信息,有助于信号分类,底层特征图具有更高的空间分辨率,保留了精确的位置信息。
通过将二者有效结合,模型能够同时兼顾对大、小不同尺度信号的检测能力;
(2)标准的Transformer注意力机制会对特征图上的所有像素点进行计算,对于仅占图像一小部分的稀疏信号而言,这会带来巨大的计算浪费。
Deformable DETR则是采用了可变形注意力机制,即不再对整个特征图进行全局计算,而是让网络根据输入动态地学习少数关键采样点的位置,并将注意力集中在这些与目标最相关的区域。
这不仅大幅降低了模型的计算复杂度和内存消耗,还可以使得注意力能够更精准地聚焦于目标区域,从而显著提升对窄带短时信号这一类小目标的检测精度。

\section{研究方案}

为实现上述研究目标,本课题计划遵循以下技术路线,分阶段进行:

% \begin{figure}[htbp]
% \centering
% \begin{minipage}{0.4\textwidth}
% \centering
% \includegraphics[width=\textwidth]{technical_route}
% \bicaption[golfer2]{}{打高尔夫球的人}{Fig.$\!$}{The person playing golf}
% \end{minipage}
% \centering
% \end{figure}

\begin{figure}[htbp]
\centering
\includegraphics[width=\textwidth]{technology roadmap}
\bicaption[{technology roadmap}]{}{技术路线图}{Fig.$\!$}{technology roadmap}
\end{figure}

% \begin{figure}[htbp]
% \centering
% \makebox[\textwidth][c]{%
%     \includegraphics[width=1.2\textwidth]{technical_route}%
% }
% \bicaption[golfer2]{}{打高尔夫球的人}{Fig.$\!$}{The person playing golf}
% \end{figure}


1. 数据集分析与预处理:
(1)对现有的真实电磁环境信号数据集进行探索性数据分析(Exploratory Data Analysis, EDA),明确其中信号制式类型的均衡性、不同信号的时宽与带宽分布、以及单个样本中的信号密度等关键特征。
(2)通过STFT将一维时域采样数据转换为二维时频谱图。在此过程中,需要仔细选择窗函数、窗长及重叠率等参数,以寻求时间和频率分辨率之间的最佳平衡。
(3)对部分样本进行可视化观察,以直观检验STFT参数的合理性,并确认是否需要对生成的时频谱图进行幅度归一化(如Min-Max归一化或Z-Score归一化)处理,以利于模型的稳定训练与快速收敛。

2. 基线模型搭建与实验:
为了科学、定量地评估本课题所提方法的有效性,将首先搭建一个基线(Baseline)模型。
考虑到YOLO系列在目标检测领域中速度与精度的良好平衡性,拟选择YOLOv11作为基线模型。
将在预处理完成的数据集上对其进行训练与测试,并详细记录其平均精度均值(mAP)、精确率(Precision)、召回率(Recall)以及检测速度(FPS)等核心性能指标,
为后续所有改进模型提供用于比较和评估的参照标准。

3. 核心模型的设计与实现:
本阶段是研究的核心,将重点围绕Deformable DETR模型展开,将其迁移至信号时频检测这一特定场景并进行适配与优化。
(1)搭建基于Deformable DETR的信号检测框架,该框架由CNN骨干网络、集成了多尺度特征融合与可变形注意力的Transformer编解码器,以及用于目标分类和边界框回归的线性预测层构成。
(2)针对性地解决多尺度输入问题,在训练流程中设计数据加载策略,将尺寸相同的时频谱图组织在同一Batch中送入网络,充分利用模型对可变尺寸输入的兼容性。
(3)将根据信号数据的特性对模型进行微调,例如调整Transformer中的对象查询数量以匹配数据集中信号的典型密度,或优化损失函数中分类损失与回归损失的权重,使模型更专注于本任务的特定挑战。

4. 综合实验与性能评估:
在完成核心模型的设计与实现后,通过多维度实验,全面、深入地验证核心模型的性能与有效性。
(1)整体性能对比:将所提出的Deformable DETR模型与基线模型(YOLOv11)在同一测试集上进行比较,分析两者在mAP、精确率、召回率等关键指标上的差异,验证核心模型的总体优越性。
(2)消融实验: 为验证模型关键组件的有效性,将设计消融研究。例如,通过对比Deformable DETR与使用单尺度特征的基础版DETR的性能,来量化多尺度特征融合(FPN思想)与可变形注意力机制对于检测窄带短时“小目标”信号的具体提升效果。
(3)定性分析: 除了定量的指标对比,还将对检测结果进行可视化分析。重点选取一些具有挑战性的样本,如包含低信噪比信号、密集重叠信号或极端长宽比信号的频谱图,直观对比不同模型在这些复杂场景下的检测效果,以展示本研究模型在鲁棒性与精确性上的优势。

5. 总结结论与论文撰写:
系统性地整理和总结所有实验数据与分析结果,凝练出本研究的核心结论,
明确所提出的信号检测模型相较于传统方法和基线模型的优势所在,并客观分析其可能存在的局限性。
在此基础上,将对未来可能的研究方向进行展望,并依据学位论文的规范要求完成毕业设计的撰写工作。



% 以下为原模板的示例内容

%% 第1章
%% !TEX root = ../main.tex

\chapter{课题背景及研究的目的与意义}

% ========== 1.1 课题背景 ==========
\section{课题背景}

随着物联网(Internet of Things, IoT)技术的迅猛发展,无线通信设备数量正呈爆炸式增长。
从智能家居、工业自动化到智慧医疗与无人驾驶,海量的无线设备与通信技术在有限的频谱资源中交织共存,构成了日益拥挤且动态多变的复杂电磁环境。
在此背景下,信号检测(Signal Detection)技术的重要性日益凸显,它不仅是提升频谱共享效率、降低信号间干扰的关键,也是保障无线通信安全、监控非法信号的必要手段。

与仅关注频段是否被占用的传统频谱感知(Spectrum Sensing, SS)不同,
信号检测旨在同时完成两项核心子任务:
(1)对环境中的信号进行精确的时频定位(即确定信号在何时占用哪些频谱资源);
(2)对定位到的信号进行准确的制式识别(即判断信号的调制类型或通信协议)。
由此可知,信号检测是实现动态频谱接入(Dynamic Spectrum Access, DSA)和保障通信可靠性的关键前提。

然而,能量检测、匹配滤波的传统频段定位方式和基于机器学习的传统信号分类方式由于其对噪声功率敏感、依赖信号先验信息以及计算复杂度高等固有局限,
难以完全适应当前非协作、高动态的复杂电磁环境。

近年来,深度学习(Deep Learning, DL)领域的突破性进展为应对上述挑战提供了全新思路。
通过将一维时域信号经过短时傅里叶变换(Short-Time Fourier Transform, STFT)转化为二维时频图,
信号感知任务得以巧妙地转化为计算机视觉领域中技术成熟的目标检测(Object Detection)问题。
这种跨域迁移方法的优势在于,
它能够借助目标检测模型实现端到端的处理,在无需精确信号或信道先验知识的条件下,自动学习信号在时频域的深层特征,从而在信号的精确定位与制式分类方面展现出巨大潜力。

% ========== 1.2 研究的目的与意义 ==========
\section{研究的目的与意义}

尽管计算机视觉领域的目标检测技术已日趋成熟,并涌现出众多先进方案,
但其在信号检测领域的应用仍处于初步探索阶段,不仅相关研究成果有限,且现有的技术迁移模式也存在若干亟待解决的问题:

1. 数据集与真实场景存在偏差:
目前,该领域的研究大多采用仿真生成或混合真实信号的数据集。
这些数据集普遍存在对现实场景的过度简化,例如信号制式与调制类型有限、信号时频密度较低,且通常忽略了信号间的重叠与干扰。
此外,现有数据集普遍假设了固定的采样率与采样时长,这与实际监测中参数多变的情况存在显著差异。
虽然这种简化有助于在研究初期快速验证算法的可行性,但却使其难以准确评估模型在真实复杂电磁环境下的性能与泛化能力。

2. 模型迁移缺乏针对性优化:
现有的研究在模型迁移方面,大多直接套用计算机视觉领域的经典模型(如Faster R-CNN),而未根据信号时频图的内在特征(相较于自然图像形态差异巨大)对模型结构进行针对性调整。
虽然近期有研究开始尝试对模型进行改进,但仍主要局限于传统的卷积神经网络(Convolutional Neural Network, CNN)架构,未能引入近年来在计算机视觉领域取得突破性进展的Transformer架构(如DETR、Deformable DETR等),其潜力有待进一步发掘。

因此,本课题旨在针对上述问题展开深入研究。研究将在技术迁移的基础上,充分考虑更为复杂多变的信号检测场景,从数据处理、模型结构设计到训练策略等环节进行系统性的优化。
其最终目标是实现目标检测技术与信号检测任务的深度融合,显著提升信号检测模型的准确率与鲁棒性。
这一技术在具体实践中将直接赋能两大关键应用:
(1)在DSA中提高信号对总体时频资源的利用率,同时降低信号之间的干扰;
(2)在复杂电磁环境中识别已授权和未授权的信号,并精确定位非法信号,从而保障无线通信的电磁空间安全。

%
%% 第2章
%% !TEX root = ../main.tex

% 中英标题:\chapter{中文标题}[英文标题]
\chapter{研究现状及分析}

\section{国内外研究现状}

建议大家仔细阅读示例文档及相关代码,理解插图的方式等。

\lipsum[1]

\section{国内外文献综述及简析}

\subsection{本硕论文题注}[Other picture example]

本硕论文题注如\figref{fig:bm}所示。

\begin{figure}[ht]
\centering
\includegraphics[width = 0.4\textwidth]{golfer}
\caption{打高尔夫球的人,硕士论文要求只用汉语}
\label{fig:bm}
\end{figure}

\subsection{并排图和子图}[Abreast-picture and Sub-picture example]
\subsubsection{并排图}[Abreast-picture example]

使用并排图时,需要注意对齐方式。默认情况是中部对齐。这里给出中部对齐、顶部对齐
、图片底部对齐三种常见方式。其中,底部对齐方式有一个很巧妙的方式,将长度比较小
的图放在左面即可。


\begin{figure}[htbp]
\centering
\begin{minipage}{0.4\textwidth}
\centering
\includegraphics[width=\textwidth]{golfer}
\bicaption[golfer2]{}{打高尔夫球的人}{Fig.$\!$}{The person playing golf}
\end{minipage}
\centering
\begin{minipage}{0.4\textwidth}
\centering
\includegraphics[width=\textwidth]{golfer}
\bicaption[golfer3]{}{打高尔夫球的人。注意,这里默认居中}{Fig.$\!$}{The person playing golf. Please note that, it is vertically center aligned by default.}
\end{minipage}
\end{figure}

\begin{figure}[htbp]
\centering
\begin{minipage}[t]{0.4\textwidth}
\centering
\includegraphics[width=\textwidth]{golfer}
\bicaption[golfer5]{}{打高尔夫球的人}{Fig.$\!$}{The person playing golf}
\end{minipage}
\centering
\begin{minipage}[t]{0.4\textwidth}
\centering
\includegraphics[width=\textwidth]{golfer}
\bicaption[golfer8]{}{打高尔夫球的人。注意,此图是顶部对齐}{Fig.$\!$}{The person playing golf. Please note that, it is vertically top aligned.}
\end{minipage}
\end{figure}

\begin{figure}[htbp]
\centering
\begin{minipage}[t]{0.4\textwidth}
\centering
\includegraphics[width=\textwidth,height=\textwidth]{golfer}
\bicaption[golfer9]{}{打高尔夫球的人。注意,此图对齐方式是图片底部对齐}{Fig.$\!$}{The person playing golf. Please note that, it is vertically bottom aligned for figure.}
\end{minipage}
\centering
\begin{minipage}[t]{0.4\textwidth}
\centering
\includegraphics[width=\textwidth]{golfer}
\bicaption[golfer6]{}{打高尔夫球的人}{Fig.$\!$}{The person playing golf}
\end{minipage}
\end{figure}

\subsubsection{子图}[Sub-picture example]

注意:子图题注也可以只用中文。规范规定“分图题置于分图之下或图题之下”,但没有给出具体的格式要求。
没有要求的另外一个说法就是“无论什么格式都不对”。
所以只有在一个图中有标注“(a),(b)”,无法使用\cs{subfigure}的情况下,使用最后一个图例中的格式设置方法,否则不要使用。
为了应对“无论什么格式都不对”,这个子图图题使用“minipage”和“description”环境,宽度,对齐方式可以按照个人喜好自由设置,是否使用双语子图图题也可以自由设置。

每页底部不要留空白。为此这里添加了无意义文字以演示。Quisque ullamcorper placerat ipsum. Cras nibh. Morbi vel justo vitae lacus tincidunt
ultrices. Lorem ipsum dolor sit amet, consectetuer adipiscing elit. In hac habitasse platea
dictumst. Integer tempus convallis augue. Etiam facilisis. Nunc elementum fermentum
wisi.

\begin{figure}[!ht]
\setlength{\subfigcapskip}{-1bp}
\centering
\begin{minipage}{\textwidth}
\centering
\subfigure{\label{golfer41}}\addtocounter{subfigure}{-2}
\subfigure[The person playing golf]{\subfigure[打高尔夫球的人~1]{\includegraphics[width=0.4\textwidth]{golfer}}}
\hspace{2em}
\subfigure{\label{golfer42}}\addtocounter{subfigure}{-2}
\subfigure[The person playing golf]{\subfigure[打高尔夫球的人~2]{\includegraphics[width=0.4\textwidth]{golfer}}}
\end{minipage}
\centering
\begin{minipage}{\textwidth}
\centering
\subfigure{\label{golfer43}}\addtocounter{subfigure}{-2}
\subfigure[The person playing golf]{\subfigure[打高尔夫球的人~3]{\includegraphics[width=0.4\textwidth]{golfer}}}
\hspace{2em}
\subfigure{\label{golfer44}}\addtocounter{subfigure}{-2}
\subfigure[The person playing golf. Here, 'hang indent' and 'center last line' are not stipulated in the regulation.]{\subfigure[打高尔夫球的人~4。注意,规范中没有明确规定要悬挂缩进、最后一行居中。]{\includegraphics[width=0.4\textwidth]{golfer}}}
\end{minipage}
\vspace{0.2em}
\bicaption[golfer4]{}{打高尔夫球的人}{Fig.$\!$}{The person playing gol}
\end{figure}

\begin{figure}[t]
  \centering
  \begin{minipage}{.7\linewidth}
    \setlength{\subfigcapskip}{-1bp}
    \centering
    \begin{minipage}{\textwidth}
      \centering
      \subfigure{\label{golfer45}}\addtocounter{subfigure}{-2}
      \subfigure[The person playing golf]{\subfigure[打高尔夫球的人~1]{\includegraphics[width=0.4\textwidth]{golfer}}}
      \hspace{4em}
      \subfigure{\label{golfer46}}\addtocounter{subfigure}{-2}
      \subfigure[The person playing golf]{\subfigure[打高尔夫球的人~2]{\includegraphics[width=0.4\textwidth]{golfer}}}
    \end{minipage}
    \vskip 0.2em
  \wuhao 注意:这里是中文图注添加位置(我工要求,图注在图题之上)。
    \vspace{0.2em}
\bicaption[golfer47]{}{打高尔夫球的人。注意,此处我工有另外一处要求,子图图题可以位于主图题之下。但由于没有明确说明位于下方具体是什么格式,所以这里不给出举例。}{Fig.$\!$}{The person playing golf. Please note that, although it is appropriate to put subfigures' captions under this caption as stipulated in regulation, but its format is not clearly stated.}
  \end{minipage}
\end{figure}

\begin{figure}[t]
\centering
% \begin{tikzpicture}
% 	\node[anchor=south west,inner sep=0] (image) at (0,0) {\includegraphics[width=0.3\textwidth]{golfer}};
% 	\begin{scope}[x={(image.south east)},y={(image.north west)}]
% 		\node at (0.3,0.5) {a)};
% 		\node at (0.8,0.2) {b)};
% 	\end{scope}
% \end{tikzpicture}
\includegraphics[width=0.3\textwidth]{golfer}
\bicaption[golfer0]{}{打高尔夫球球的人(博士论文双语题注)}{Fig.$\!$}{The person playing golf (Doctoral thesis)}
\vskip -0.4em
 \hspace{2em}
\begin{minipage}[t]{0.3\textwidth}
\wuhao \setlist[description]{font=\normalfont}
	\begin{description}
		\item[(a)]子图图题
		\item[(a)]Subfigure caption
	\end{description}
 \end{minipage}
 \hspace{2em}
 \begin{minipage}[t]{0.3\textwidth}
\wuhao \setlist[description]{font=\normalfont}
	\begin{description}
		\item[(b)]子图图题
		\item[(b)]Subfigure caption
	\end{description}
\end{minipage}
\end{figure}

\begin{figure}[!ht]
	\centering
	\begin{sideways}
		\begin{minipage}{\textheight}
			\centering
			\fbox{\includegraphics[width=0.2\textwidth]{golfer}}
			\fbox{\includegraphics[width=0.2\textwidth]{golfer}}
			\fbox{\includegraphics[width=0.2\textwidth]{golfer}}
			\fbox{\includegraphics[width=0.2\textwidth]{golfer}}
			\fbox{\includegraphics[width=0.2\textwidth]{golfer}}
			\fbox{\includegraphics[width=0.2\textwidth]{golfer}}
			\fbox{\includegraphics[width=0.2\textwidth]{golfer}}
\bicaption[golfer7]{}{打高尔夫球的人(非规范要求)}{Fig.$\!$}{The person playing golf (Not stated in the regulation)}
		\end{minipage}
	\end{sideways}
\end{figure}

\clearpage

如果不想让图片浮动到下一章节,那么在此处使用\cs{clearpage}命令。

\subsection{如何做出符合规范的漂亮的图}

关于作图工具在后文\ref{drawtool}中给出一些作图工具的介绍,此处不多言。
此处以R语言和Tikz为例说明如何做出符合规范的图。

\subsubsection{Tikz作图}

使用Tikz作图核心思想是把格式、主题、样式与内容分离,定义在全局中。
注意字体设置可以有两种选择,如果字少,用五号字,字多用小五。
使用Tikz作图不会出现字体问题,字体会自动与正文一致。

\subsubsection{R作图}

R是一种极具有代表性的典型的作图工具,应用广泛。
与Tikz图不同,R作图分两种情况:(1)可以转换为Tikz码;(2)不可转换为Tikz码。
第一种情况图形简单,图形中不含有很多数据点,使用R语言中的Tikz包即可。
第二种情况是图形复杂,含有海量数据点,这时候不要转成Tikz矢量图,这会使得论文体积巨大。
推荐使用pdf或png非矢量图形。
使用非矢量图形时要注意选择好字号(五号或小五),和字体(宋体、新罗马)然后选择生成图形大小,注意此时在正文中使用\cs{includegraphics}命令导入时,不要像导入矢量图那样控制图形大小,使用图形的原本的
宽度和高度,这样就确保了非矢量图形中的文字与正文一致了。

为了控制\hitszthesis\ 的大小,此处不给出具体举例。

\subsubsection{专业绘图工具}
\label{drawtool}

推荐使用tikz包,使用tikz源码绘图的好处是,图片中的字体与正文中的字体一致。具体如
何使用tikz绘图不属于模板范畴。

tikz适合用来画不需要大量实验数据支撑示意图。但R语言等专业绘图工具具有画出各种、
专业、复杂的数据图。R语言中有tikz包,能自动生成tikz码,这样tikz几乎无所不能。
对于排版有极致追求的小伙伴,可以参考
\href{http://www.texample.net/tikz/resources/}{http://www.texample.net/tikz/resources/}
中所列工具,几乎所有作图软件所作的图形都可转成tikz,然后可以自由的在tikz中修改
图中内容,定义字体等等。实现前文窝工规范中要求的图中字体的一致性的终极目标。


%
%% 第3章
%% !TEX root = ../main.tex

% 中英标题:\chapter{中文标题}[英文标题]
\chapter{主要研究内容及研究方案}

% ========== 3.1 研究内容 ==========
\section{研究内容}

\subsection{针对不同尺度信号数据的处理方法研究}
在在真实的电磁监测场景中,采样率、采样时长等采集参数的多样性,对数据处理提出了关键挑战。
为保证神经网络能够有效学习信号的内在模式,进行STFT时,必须设定固定的频率分辨率与时间分辨率。
然而,这一约束将不可避免地导致采用不同采样参数的样本在STFT后生成尺寸大小不一的二维时频谱图。

传统的计算机视觉模型,特别是包含全连接层的网络,通常要求输入图像具有固定的尺寸。
尽管可通过插值缩放或统一尺寸填充(Padding)来强制对齐,但这两种方法均不适用于本任务:
前者会严重扭曲信号固有的时频结构,破坏其物理意义;
后者在样本尺寸差异悬殊时,会引入大量无信息的零值填充,不仅造成计算资源的极大浪费,还可能干扰模型的有效特征学习。

针对此问题,本研究拟对目标检测领域的多尺度训练(Multiscale Training)策略进行创新性改造。
不同于在视觉任务中将其作为一种可选的数据增强手段,本课题将多尺度处理升格为一种应对异构数据源的必要核心机制。
具体而言,本研究将充分利用所选DETR模型(由CNN骨干网络、Transformer及预测层构成)对输入尺寸无严格限制的结构优势,
在训练流程中设计一种自适应批处理(Adaptive Batching)策略,即将尺寸相同的STFT数据动态地组织成批次(Batch)送入网络。
该策略旨在使模型能够稳健、高效地处理来源多样化的真实信号数据,从根本上解决异构尺度输入的难题。

\subsection{基于Transformer检测头的信号检测模型研究}
检测头(Detection Head),即负责从深度特征图中解码出目标位置和类别的网络末端部分,是决定检测器性能的关键组件。
主流的检测头可依据其结构特性分为三种范式,且各有优劣:
(1)全连接检测头:其优点在于可以人为设定固定数量的预测输出,无需复杂的后处理。但其结构要求输入特征图必须为固定尺寸,这与前述的多尺度输入策略相悖,缺乏灵活性。
(2)全卷积检测头:该结构能够灵活地适应不同尺寸的输入,但其输出的预测框数量不固定,与输入尺寸相关,因此必须依赖NMS等复杂的后处理算法来滤除冗余的预测框。
(3)Transformer检测头:该结构结合了前两者的优点:既能处理可变尺寸的输入,又能通过其内部的对象查询(Object Queries)机制输出固定数量的预测结果,从而省去了复杂的后处理环节,构建了真正意义上的端到端检测流程。

更重要的是,Transformer架构的核心——自注意力机制(Self-Attention),赋予了模型强大的全局依赖建模能力,这对于捕捉信号间的谐波关系、跳频模式等全局性特征具有潜在优势。
然而,从时频谱图的视觉先验来看,信号的形态多为局部性较强的矩形或条状。
全局注意力机制是否会引入不必要的计算冗余,以及其对于信号局部特征的捕捉是否优于卷积网络,是一个需要通过实验深入探究的关键问题。
因此,本研究将重点构建并评估基于Transformer的检测头,以验证其在信号检测任务中的实际效能与适用性。

\subsection{针对窄带短时信号的模型优化策略研究}
在时频谱图中,窄带短时信号(如猝发信号、短时通信等)表现为尺寸微小、持续时间短的“小目标”,这对检测模型提出了极高的要求。
标准的检测模型由于特征图分辨率较低、感受野过大等问题,在处理这类小目标时常出现漏检的现象。
为攻克这一难题,本研究拟引入Deformable DETR模型进行优化:
(1)Deformable DETR借鉴了FPN的思想,融合了骨干网络在不同阶段输出的多尺度特征图。
高层特征图包含丰富的语义信息,有助于信号分类,底层特征图具有更高的空间分辨率,保留了精确的位置信息。
通过将二者有效结合,模型能够同时兼顾对大、小不同尺度信号的检测能力;
(2)标准的Transformer注意力机制会对特征图上的所有像素点进行计算,对于仅占图像一小部分的稀疏信号而言,这会带来巨大的计算浪费。
Deformable DETR则是采用了可变形注意力机制,即不再对整个特征图进行全局计算,而是让网络根据输入动态地学习少数关键采样点的位置,并将注意力集中在这些与目标最相关的区域。
这不仅大幅降低了模型的计算复杂度和内存消耗,还可以使得注意力能够更精准地聚焦于目标区域,从而显著提升对窄带短时信号这一类小目标的检测精度。

\section{研究方案}

为实现上述研究目标,本课题计划遵循以下技术路线,分阶段进行:

% \begin{figure}[htbp]
% \centering
% \begin{minipage}{0.4\textwidth}
% \centering
% \includegraphics[width=\textwidth]{technical_route}
% \bicaption[golfer2]{}{打高尔夫球的人}{Fig.$\!$}{The person playing golf}
% \end{minipage}
% \centering
% \end{figure}

\begin{figure}[htbp]
\centering
\includegraphics[width=\textwidth]{technology roadmap}
\bicaption[{technology roadmap}]{}{技术路线图}{Fig.$\!$}{technology roadmap}
\end{figure}

% \begin{figure}[htbp]
% \centering
% \makebox[\textwidth][c]{%
%     \includegraphics[width=1.2\textwidth]{technical_route}%
% }
% \bicaption[golfer2]{}{打高尔夫球的人}{Fig.$\!$}{The person playing golf}
% \end{figure}


1. 数据集分析与预处理:
(1)对现有的真实电磁环境信号数据集进行探索性数据分析(Exploratory Data Analysis, EDA),明确其中信号制式类型的均衡性、不同信号的时宽与带宽分布、以及单个样本中的信号密度等关键特征。
(2)通过STFT将一维时域采样数据转换为二维时频谱图。在此过程中,需要仔细选择窗函数、窗长及重叠率等参数,以寻求时间和频率分辨率之间的最佳平衡。
(3)对部分样本进行可视化观察,以直观检验STFT参数的合理性,并确认是否需要对生成的时频谱图进行幅度归一化(如Min-Max归一化或Z-Score归一化)处理,以利于模型的稳定训练与快速收敛。

2. 基线模型搭建与实验:
为了科学、定量地评估本课题所提方法的有效性,将首先搭建一个基线(Baseline)模型。
考虑到YOLO系列在目标检测领域中速度与精度的良好平衡性,拟选择YOLOv11作为基线模型。
将在预处理完成的数据集上对其进行训练与测试,并详细记录其平均精度均值(mAP)、精确率(Precision)、召回率(Recall)以及检测速度(FPS)等核心性能指标,
为后续所有改进模型提供用于比较和评估的参照标准。

3. 核心模型的设计与实现:
本阶段是研究的核心,将重点围绕Deformable DETR模型展开,将其迁移至信号时频检测这一特定场景并进行适配与优化。
(1)搭建基于Deformable DETR的信号检测框架,该框架由CNN骨干网络、集成了多尺度特征融合与可变形注意力的Transformer编解码器,以及用于目标分类和边界框回归的线性预测层构成。
(2)针对性地解决多尺度输入问题,在训练流程中设计数据加载策略,将尺寸相同的时频谱图组织在同一Batch中送入网络,充分利用模型对可变尺寸输入的兼容性。
(3)将根据信号数据的特性对模型进行微调,例如调整Transformer中的对象查询数量以匹配数据集中信号的典型密度,或优化损失函数中分类损失与回归损失的权重,使模型更专注于本任务的特定挑战。

4. 综合实验与性能评估:
在完成核心模型的设计与实现后,通过多维度实验,全面、深入地验证核心模型的性能与有效性。
(1)整体性能对比:将所提出的Deformable DETR模型与基线模型(YOLOv11)在同一测试集上进行比较,分析两者在mAP、精确率、召回率等关键指标上的差异,验证核心模型的总体优越性。
(2)消融实验: 为验证模型关键组件的有效性,将设计消融研究。例如,通过对比Deformable DETR与使用单尺度特征的基础版DETR的性能,来量化多尺度特征融合(FPN思想)与可变形注意力机制对于检测窄带短时“小目标”信号的具体提升效果。
(3)定性分析: 除了定量的指标对比,还将对检测结果进行可视化分析。重点选取一些具有挑战性的样本,如包含低信噪比信号、密集重叠信号或极端长宽比信号的频谱图,直观对比不同模型在这些复杂场景下的检测效果,以展示本研究模型在鲁棒性与精确性上的优势。

5. 总结结论与论文撰写:
系统性地整理和总结所有实验数据与分析结果,凝练出本研究的核心结论,
明确所提出的信号检测模型相较于传统方法和基线模型的优势所在,并客观分析其可能存在的局限性。
在此基础上,将对未来可能的研究方向进行展望,并依据学位论文的规范要求完成毕业设计的撰写工作。


%
%% 第4章
%% !TEX root = ../main.tex

\chapter{进度安排及预期目标}

% ========== 4.1 进度安排 ==========
\section{进度安排}
本课题的研究时间为2025年9月开始至2026年5月,根据之前的项目经验,结合自身的学习科研能力,针对本课题研究目标,特制定以下研究进度计划:

1. 2025年9月至2025年10月:
(1)深入研读相关领域的前沿文献,完善并最终确定详细的技术实现方案;
(2)对信号数据集进行全面的统计性分析,完成STFT转换、数据清洗、归一化及标注格式统一等所有预处理工作;
(3)完成深度学习实验环境的搭建、配置与调试,并验证数据加载与基础训练流程的稳定性。

2. 2025年11月至2026年3月:
(1)实现YOLOv11基线模型与基于Deformable DETR的核心检测模型;
(2)解决多尺度时频谱图的输入与批处理问题,
(3)在数据集上完成模型的完整训练、调优与多维度性能评估,获取全面的实验数据,并进行系统性的对比分析,得出研究结论。

3. 2026年4月至2026年5月:
(1)根据已有的实验结果,撰写毕业论文,准备毕业设计答辩。

% ========== 4.2 预期目标 ==========
\section{预期目标}
本课题旨在通过上述研究,达成以下具体目标:

1. 深入分析真实电磁环境信号数据集的关键特性,完成科学的STFT数据预处理。
最终构建一套能够支撑异构尺度输入训练的、规范化的实验与验证数据集,为后续研究提供坚实的数据基础。

2. 成功复现并将在信号时频谱图检测任务上适配、调优一个高性能的YOLOv11模型。
获取其在各项关键指标(精确率、召回率、mAP等)上的可靠性能数据,为衡量本课题核心模型的先进性提供一个强有力的参照基准。

3. 成功将先进的端到端目标检测框架Deformable DETR迁移并优化,使其深度适配信号时频联合检测任务。
通过充分的实验对比,验证该模型相较于传统CNN基线模型,在复杂电磁环境下,特别是在处理密集、微小及重叠信号时的鲁棒性与性能优越性。

4. 针对真实场景中STFT时频谱图尺寸不一的核心工程挑战,成功设计并实现一种创新的多尺度自适应训练策略。
该策略旨在从根本上解决异构数据输入的难题,避免因传统处理方式(插值、填充)导致的特征失真与资源浪费,提升模型的工程实用性。
%
%% 第5章
%% !TEX root = ../main.tex

\chapter{已具备和所需的条件和经费}

% ========== 5.1 已具备的条件 ==========
\section{已具备的条件}

\subsection{实验室已具备条件}
本研究依托实验室现有的高性能计算服务器。

该服务器配备了128GB系统内存与NVIDIA RTX 4090高性能图形处理器(GPU),其强大的计算能力与充足的显存资源,能够为本课题中大规模数据集的处理及深度学习模型的训练提供充分的硬件支持。
且服务器已部署了稳定、完善的科研环境,包括Ubuntu操作系统、Python编程环境以及PyTorch等主流深度学习框架。相关开发库与依赖项均已配置完毕,可确保研究工作的顺利开展。

服务器已部署了稳定、完善的科研环境,包括Ubuntu操作系统、Python编程环境以及PyTorch等主流深度学习框架。相关开发库与依赖项均已配置完毕,可确保研究工作的顺利开展。

鉴于本课题的核心工作为算法设计与软件实现,且所需硬件与软件条件均已具备,因此本研究无需申请额外经费支持,现有资源可完全保障课题的顺利完成。

\subsection{实验室经费保障}
本课题主要通过编写代码完成,同时训练所需的硬件设备均以具备,故无需经费支持。

% ========== 5.2 所需的条件 ==========
\section{所需的条件}
所需条件已经在5.1中详细列出并已具备,无需另外条件和经费。
%
%% 第6章
%% !TEX root = ../main.tex

\chapter{预计困难及解决方案说明}

% ========== 6.1 技术难点与预计困难 ==========
\section{技术难点与预计困难}

1. 数据的异构性与复杂性:
本研究所采用的数据集包含由不同采样参数导致的异构尺寸时频谱图,如何设计高效的数据加载与批处理机制以适应这种内在的异构性,是一个关键的工程挑战。
此外,数据集中信号类型多样、时频密集且存在相互重叠,这对模型的特征表征与精细分辨能力提出了极高的要求。

2. 模型的高复杂性与超参数调优:
本研究中超参数调优的复杂度较高,涉及模型结构参数(如模块层数)、优化器与学习率调度器参数(如学习率、Warmup轮数)、以及损失函数各部分权重系数等众多变量。
不恰当的超参数配置极易导致模型不收敛或陷入局部最优,因此,高效的超参数寻优策略至关重要。

3. 微小信号检测的瓶颈:
在时频谱图中,大量关键信号(如猝发信号)呈现为时宽与带宽极窄的“小目标”。
此类目标在经过骨干网络的多层下采样后,其特征信息极易被稀释或丢失,是目标检测领域的公认技术瓶颈。
如何有效克服这一瓶颈,确保模型对微小信号的检出率与定位精度,是衡量本研究算法先进性的核心指标。

4. 训练开销与显存瓶颈:
引入Transformer架构将不可避免地带来计算量与显存占用的显著增长。
在有限的硬件资源下,为防止显存溢出而过度减小批处理大小(Batch Size),不仅会严重降低训练效率,还可能损害模型的收敛性能与最终精度。

% ========== 6.2 解决方案 ==========
\section{解决方案}

针对上述可能遇到的困难,本课题预先制定了如下应对策略和解决方案:

1. 将设计并实现一个定制化的数据整理函数(Collate Function)。
该函数能够在构建每个训练批次时,动态地将尺寸相同或相近的样本聚合,从而在不破坏原始数据结构的前提下,实现对异构数据的兼容与高效处理。

2. 将采用迁移学习范式,加载在大型公开数据集上预训练的模型权重作为优化起点,以加速收敛并提升性能。
同时,将借鉴相关领域顶级会议论文中的成熟训练策略与超参数配置,构建一个可靠的初始设定。
采用控制变量法进行系统性的超参数调优,并密切监控训练过程中的损失曲线与验证集性能。
辅以早停(Early Stopping)策略避免不必要的计算开销,并通过对模型输出结果的可视化诊断,为针对性的结构与损失函数调整提供直观依据。

3. 充分利用Deformable DETR的多尺度特征融合能力,确保来自底层网络的高分辨率特征被有效传递至检测头。
发挥可变形注意力机制的优势,引导模型将计算资源精准聚焦于微小信号所在的关键区域。

4. 将优先采用混合精度训练(Automatic Mixed Precision, AMP),在几乎不损失模型精度的前提下,大幅降低显存占用并提升训练速度。
若批大小依然受限,将采用梯度累积技术,以时间换空间,在不增加显存消耗的情况下实现等效的大批次训练效果。
在必要时,可利用实验室的多GPU资源,通过数据并行(Data Parallelism)等分布式训练框架,从根本上扩展可用计算与显存资源,突破单卡硬件瓶颈。

% 开始写正文之后的部分
\backmatter

%%%% \begin{本科书序} %%%% 这是一个假的环境,本科请用这里的内容

% % 结论
% \input{back/conclusion}

% % 参考文献
\bibliographystyle{hitszthesis}
\bibliography{reference}

% % 发表文章
% \input{back/publications}

% % 授权
% \authorization

% % 授权页为扫描的PDF文件(scan.pdf),与上面的命令互斥
% % \authorization[scan.pdf]

% % 致谢
% \input{back/acknowledgements}

% % 附录
% % 设置附录部分只包含页眉
% % \SetAppendixWithOnlyHeadings
% % 设置附录部分页码从1开始编号的命令在<back/appendix01.tex>里
% \begin{appendix}
%   \input{back/appendix01}
%   \input{back/appendix02}
%   \input{back/appendix03}
% \end{appendix}

%%%% \end{本科书序}


% 结束文档撰写
\end{document}
%%=============================================

% Local Variables:
% TeX-engine: xetex
% End:
