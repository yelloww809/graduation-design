% !TEX root = ../main.tex

\chapter{进度安排及预期目标}

% ========== 4.1 进度安排 ==========
\section{进度安排}
本课题的研究时间为2025年9月开始至2026年5月,根据之前的项目经验,结合自身的学习科研能力,针对本课题研究目标,特制定以下研究进度计划:

1. 2025年9月至2025年10月:
(1)深入研读相关领域的前沿文献,完善并最终确定详细的技术实现方案;
(2)对信号数据集进行全面的统计性分析,完成STFT转换、数据清洗、归一化及标注格式统一等所有预处理工作;
(3)完成深度学习实验环境的搭建、配置与调试,并验证数据加载与基础训练流程的稳定性。

2. 2025年11月至2026年3月:
(1)实现YOLOv11基线模型与基于Deformable DETR的核心检测模型;
(2)解决多尺度时频谱图的输入与批处理问题,
(3)在数据集上完成模型的完整训练、调优与多维度性能评估,获取全面的实验数据,并进行系统性的对比分析,得出研究结论。

3. 2026年4月至2026年5月:
(1)根据已有的实验结果,撰写毕业论文,准备毕业设计答辩。

% ========== 4.2 预期目标 ==========
\section{预期目标}
本课题旨在通过上述研究,达成以下具体目标:

1. 深入分析真实电磁环境信号数据集的关键特性,完成科学的STFT数据预处理。
最终构建一套能够支撑异构尺度输入训练的、规范化的实验与验证数据集,为后续研究提供坚实的数据基础。

2. 成功复现并将在信号时频谱图检测任务上适配、调优一个高性能的YOLOv11模型。
获取其在各项关键指标(精确率、召回率、mAP等)上的可靠性能数据,为衡量本课题核心模型的先进性提供一个强有力的参照基准。

3. 成功将先进的端到端目标检测框架Deformable DETR迁移并优化,使其深度适配信号时频联合检测任务。
通过充分的实验对比,验证该模型相较于传统CNN基线模型,在复杂电磁环境下,特别是在处理密集、微小及重叠信号时的鲁棒性与性能优越性。

4. 针对真实场景中STFT时频谱图尺寸不一的核心工程挑战,成功设计并实现一种创新的多尺度自适应训练策略。
该策略旨在从根本上解决异构数据输入的难题,避免因传统处理方式(插值、填充)导致的特征失真与资源浪费,提升模型的工程实用性。