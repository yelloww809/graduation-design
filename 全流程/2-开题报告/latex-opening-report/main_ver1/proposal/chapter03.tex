% !TEX root = ../main.tex

% 中英标题:\chapter{中文标题}[英文标题]
\chapter{主要研究内容及研究方案}

% ========== 3.1 研究内容 ==========
\section{研究内容}

\subsection{针对不同尺度信号数据的处理方法研究}
在在真实的电磁监测场景中,采样率、采样时长等采集参数的多样性,对数据处理提出了关键挑战。
为保证神经网络能够有效学习信号的内在模式,进行STFT时,必须设定固定的频率分辨率与时间分辨率。
然而,这一约束将不可避免地导致采用不同采样参数的样本在STFT后生成尺寸大小不一的二维时频谱图。

传统的计算机视觉模型,特别是包含全连接层的网络,通常要求输入图像具有固定的尺寸。
尽管可通过插值缩放或统一尺寸填充(Padding)来强制对齐,但这两种方法均不适用于本任务:
前者会严重扭曲信号固有的时频结构,破坏其物理意义;
后者在样本尺寸差异悬殊时,会引入大量无信息的零值填充,不仅造成计算资源的极大浪费,还可能干扰模型的有效特征学习。

针对此问题,本研究拟对目标检测领域的多尺度训练(Multiscale Training)策略进行创新性改造。
不同于在视觉任务中将其作为一种可选的数据增强手段,本课题将多尺度处理升格为一种应对异构数据源的必要核心机制。
具体而言,本研究将充分利用所选DETR模型(由CNN骨干网络、Transformer及预测层构成)对输入尺寸无严格限制的结构优势,
在训练流程中设计一种自适应批处理(Adaptive Batching)策略,即将尺寸相同的STFT数据动态地组织成批次(Batch)送入网络。
该策略旨在使模型能够稳健、高效地处理来源多样化的真实信号数据,从根本上解决异构尺度输入的难题。

\subsection{基于Transformer检测头的信号检测模型研究}
检测头(Detection Head),即负责从深度特征图中解码出目标位置和类别的网络末端部分,是决定检测器性能的关键组件。
主流的检测头可依据其结构特性分为三种范式,且各有优劣:
(1)全连接检测头:其优点在于可以人为设定固定数量的预测输出,无需复杂的后处理。但其结构要求输入特征图必须为固定尺寸,这与前述的多尺度输入策略相悖,缺乏灵活性。
(2)全卷积检测头:该结构能够灵活地适应不同尺寸的输入,但其输出的预测框数量不固定,与输入尺寸相关,因此必须依赖NMS等复杂的后处理算法来滤除冗余的预测框。
(3)Transformer检测头:该结构结合了前两者的优点:既能处理可变尺寸的输入,又能通过其内部的对象查询(Object Queries)机制输出固定数量的预测结果,从而省去了复杂的后处理环节,构建了真正意义上的端到端检测流程。

更重要的是,Transformer架构的核心——自注意力机制(Self-Attention),赋予了模型强大的全局依赖建模能力,这对于捕捉信号间的谐波关系、跳频模式等全局性特征具有潜在优势。
然而,从时频谱图的视觉先验来看,信号的形态多为局部性较强的矩形或条状。
全局注意力机制是否会引入不必要的计算冗余,以及其对于信号局部特征的捕捉是否优于卷积网络,是一个需要通过实验深入探究的关键问题。
因此,本研究将重点构建并评估基于Transformer的检测头,以验证其在信号检测任务中的实际效能与适用性。

\subsection{针对窄带短时信号的模型优化策略研究}
在时频谱图中,窄带短时信号(如猝发信号、短时通信等)表现为尺寸微小、持续时间短的“小目标”,这对检测模型提出了极高的要求。
标准的检测模型由于特征图分辨率较低、感受野过大等问题,在处理这类小目标时常出现漏检的现象。
为攻克这一难题,本研究拟引入Deformable DETR模型进行优化:
(1)Deformable DETR借鉴了FPN的思想,融合了骨干网络在不同阶段输出的多尺度特征图。
高层特征图包含丰富的语义信息,有助于信号分类,底层特征图具有更高的空间分辨率,保留了精确的位置信息。
通过将二者有效结合,模型能够同时兼顾对大、小不同尺度信号的检测能力;
(2)标准的Transformer注意力机制会对特征图上的所有像素点进行计算,对于仅占图像一小部分的稀疏信号而言,这会带来巨大的计算浪费。
Deformable DETR则是采用了可变形注意力机制,即不再对整个特征图进行全局计算,而是让网络根据输入动态地学习少数关键采样点的位置,并将注意力集中在这些与目标最相关的区域。
这不仅大幅降低了模型的计算复杂度和内存消耗,还可以使得注意力能够更精准地聚焦于目标区域,从而显著提升对窄带短时信号这一类小目标的检测精度。

\section{研究方案}

为实现上述研究目标,本课题计划遵循以下技术路线,分阶段进行:

% \begin{figure}[htbp]
% \centering
% \begin{minipage}{0.4\textwidth}
% \centering
% \includegraphics[width=\textwidth]{technical_route}
% \bicaption[golfer2]{}{打高尔夫球的人}{Fig.$\!$}{The person playing golf}
% \end{minipage}
% \centering
% \end{figure}

\begin{figure}[htbp]
\centering
\includegraphics[width=\textwidth]{technology roadmap}
\bicaption[{technology roadmap}]{}{技术路线图}{Fig.$\!$}{technology roadmap}
\end{figure}

% \begin{figure}[htbp]
% \centering
% \makebox[\textwidth][c]{%
%     \includegraphics[width=1.2\textwidth]{technical_route}%
% }
% \bicaption[golfer2]{}{打高尔夫球的人}{Fig.$\!$}{The person playing golf}
% \end{figure}


1. 数据集分析与预处理:
(1)对现有的真实电磁环境信号数据集进行探索性数据分析(Exploratory Data Analysis, EDA),明确其中信号制式类型的均衡性、不同信号的时宽与带宽分布、以及单个样本中的信号密度等关键特征。
(2)通过STFT将一维时域采样数据转换为二维时频谱图。在此过程中,需要仔细选择窗函数、窗长及重叠率等参数,以寻求时间和频率分辨率之间的最佳平衡。
(3)对部分样本进行可视化观察,以直观检验STFT参数的合理性,并确认是否需要对生成的时频谱图进行幅度归一化(如Min-Max归一化或Z-Score归一化)处理,以利于模型的稳定训练与快速收敛。

2. 基线模型搭建与实验:
为了科学、定量地评估本课题所提方法的有效性,将首先搭建一个基线(Baseline)模型。
考虑到YOLO系列在目标检测领域中速度与精度的良好平衡性,拟选择YOLOv11作为基线模型。
将在预处理完成的数据集上对其进行训练与测试,并详细记录其平均精度均值(mAP)、精确率(Precision)、召回率(Recall)以及检测速度(FPS)等核心性能指标,
为后续所有改进模型提供用于比较和评估的参照标准。

3. 核心模型的设计与实现:
本阶段是研究的核心,将重点围绕Deformable DETR模型展开,将其迁移至信号时频检测这一特定场景并进行适配与优化。
(1)搭建基于Deformable DETR的信号检测框架,该框架由CNN骨干网络、集成了多尺度特征融合与可变形注意力的Transformer编解码器,以及用于目标分类和边界框回归的线性预测层构成。
(2)针对性地解决多尺度输入问题,在训练流程中设计数据加载策略,将尺寸相同的时频谱图组织在同一Batch中送入网络,充分利用模型对可变尺寸输入的兼容性。
(3)将根据信号数据的特性对模型进行微调,例如调整Transformer中的对象查询数量以匹配数据集中信号的典型密度,或优化损失函数中分类损失与回归损失的权重,使模型更专注于本任务的特定挑战。

4. 综合实验与性能评估:
在完成核心模型的设计与实现后,通过多维度实验,全面、深入地验证核心模型的性能与有效性。
(1)整体性能对比:将所提出的Deformable DETR模型与基线模型(YOLOv11)在同一测试集上进行比较,分析两者在mAP、精确率、召回率等关键指标上的差异,验证核心模型的总体优越性。
(2)消融实验: 为验证模型关键组件的有效性,将设计消融研究。例如,通过对比Deformable DETR与使用单尺度特征的基础版DETR的性能,来量化多尺度特征融合(FPN思想)与可变形注意力机制对于检测窄带短时“小目标”信号的具体提升效果。
(3)定性分析: 除了定量的指标对比,还将对检测结果进行可视化分析。重点选取一些具有挑战性的样本,如包含低信噪比信号、密集重叠信号或极端长宽比信号的频谱图,直观对比不同模型在这些复杂场景下的检测效果,以展示本研究模型在鲁棒性与精确性上的优势。

5. 总结结论与论文撰写:
系统性地整理和总结所有实验数据与分析结果,凝练出本研究的核心结论,
明确所提出的信号检测模型相较于传统方法和基线模型的优势所在,并客观分析其可能存在的局限性。
在此基础上,将对未来可能的研究方向进行展望,并依据学位论文的规范要求完成毕业设计的撰写工作。

