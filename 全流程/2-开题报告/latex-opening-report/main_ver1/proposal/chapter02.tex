% !TEX root = ../main.tex

\chapter{研究现状及分析}

% ========== 2.1 国内外研究现状 ==========
\section{国内外研究现状}

在无线通信领域,对宽带频谱中纷繁复杂的信号进行精确、快速的检测与识别,是现代频谱管理、认知无线电及电子对抗等应用的核心前提。
针对这一目标,国内外的研究历程可清晰地划分为两个主要阶段:传统信号处理方法阶段与基于深度学习的智能感知阶段。

在传统方法阶段,信号的“定位”与“分类”遵循着两条相互独立的技术路径。
一方面,信号定位技术从一维的能量检测、匹配滤波,发展到基于STFT的二维时频谱图分析,
但这些方法普遍存在泛化能力差、抗噪声性能弱的瓶颈,且无法识别信号的具体制式。
另一方面,信号分类技术无论是采用基于专家特征的传统机器学习,还是早期的卷积神经网络,
虽然能够识别信号的调制类型,却无法提供其精确的时频位置信息。
这种技术上的分离导致两条路径始终未能有效融合。

深度学习目标检测技术的崛起,为实现定位与分类的“一体化”提供了全新范式。
该技术从经典的Faster R-CNN、YOLO等卷积网络模型,演进至以DETR及其改进型为代表的、更为先进的端到端Transformer架构,为信号检测任务提供了强大的理论与模型基础。
受此启发,研究者们开始将目标检测框架应用于信号时频谱图,成功地将信号感知任务重构为视觉识别问题,并验证了此思路的可行性。

% ========== 2.2 国内外文献综述及简析 ==========
\section{国内外文献综述及简析}

\subsection{传统信号定位与识别方法研究现状}

\subsubsection{传统信号定位方式}
在信号处理的早期阶段,信号定位主要依赖于对一维时域数据的分析,其中能量检测(Energy Detection)与匹配滤波(Matched Filtering)是最具代表性的技术。

能量检测作为一种基础的非相干检测方法,通过在特定频段内对信号能量进行积分并与预设门限比较来判断信号是否存在。
Quan等人的研究\cite{4533213}为认知无线电引入了多频段联合能量检测技术,以识别频谱空洞。
然而,这类将宽带频谱划分为多个独立窄带进行检测的方法会导致频谱分析的粒度较粗,频谱利用效率不高;
Bkassiny等人\cite{6204008}则结合了盲能量检测与循环平稳特征,并辅以无监督聚类进行分类,但这增加了系统的复杂性。

匹配滤波则是一种相干检测方法,其核心思想是通过将接收信号与已知信号模板进行互相关运算来实现信号检测。
Bao等人的工作\cite{6488847}利用信号与噪声在统计分布上存在差异这一先验信息,通过直方图分析自适应地寻找阈值,以定位占用频段。
然而,然而,匹配滤波方法存在对先验知识强依赖的固有缺陷,即检测方必须预先精确掌握待测信号的完整信息(如波形、调制方式等),这使其在非合作通信、盲信号检测等场景下适用性严重受限。
此外,当需要检测多种不同类型的信号时,必须设计相应的多个匹配滤波器,导致系统实现复杂度高、灵活性差。

为了克服一维分析的局限性,研究人员引入了时频分析工具,其中短时傅里叶变换是应用最广泛的一种。
通过STFT,一维的时序信号可以被转换为二维的时频谱图(Spectrogram),从而可以直观地展示信号频率随时间的变化情况。
这一转变为研究者借鉴计算机视觉领域中成熟技术进行信号分析提供了可能性。

在对于二维时频谱图的分析中,基于数学形态学(Mathematical Morphology)的处理方法成为一个研究热点。
该方法利用腐蚀、膨胀等一系列形态学操作,对时频谱图进行去噪、边缘增强和目标提取。
例如,Mankun等人\cite{4297156}利用图像增强和数学形态学技术对跳频信号的时频谱图进行分析,有效地提取了信号的轮廓和跳变模式;
Phonsri等人\cite{7428185}进一步结合了计算机视觉与双向神经网络,用于从含噪的时频谱图中提取通信信号。
然而,这些方法的缺点也十分明显:
首先,形态学操作对噪声和干扰非常敏感,且其参数(如结构元素的尺寸和形状)需要根据信号特征进行人工设计,缺乏自适应性,泛化能力较差;
更重要的是,这些方法主要集中于“定位”,即找出信号在时频谱图中的位置和形状,但无法在此基础上进一步提供信号的调制类型、通信制式等深层语义信息。

随着深度学习技术的发展,一些研究开始尝试利用神经网络来处理频域数据。
例如,Huang等人\cite{9121253}采用全卷积网络(Fully Convolutional Network, FCN)对宽带功率谱进行分析,以实现载波信号的检测。
这种方法虽然在信号的二维时频结构信息基础上利用了深度学习强大的特征提取能力,提升了频率定位的准确性,但仍未能解决时频联合定位的根本问题。

\subsubsection{传统信号分类方式}
在深度学习普及之前,基于传统机器学习的信号分类方法是主流。
这类方法通常遵循“特征工程 + 分类器”的范式,即研究者需要手动设计并提取信号的高阶累积量、循环谱特征、小波变换特征等专家特征,再将这些特征输入到支持向量机(Support Vector Machine, SVM)、决策树等分类器中进行训练和分类。
例如,在2008年\cite{4493784}和2016年\cite{7907462}的两项研究中,学者们分别利用小波特征和新颖的统计特征,结合SVM分类器,对数字信号进行了自动调制识别。
这类方法在特定数据集上取得了不错的分类效果,但其核心缺陷在于严重依赖专家知识进行特征设计,导致特征的泛化能力和鲁棒性有限。
更关键的是,这些方法的设计目标是为信号片段赋予类别标签,完全无法提供信号在时频域中的精确位置,从而无法实现定位与分类的一体化。

深度学习,特别是卷积神经网络的引入,为信号分类带来了突破,它能够自动从原始数据中学习特征,避免了繁琐的手动特征工程。
Bitar等人\cite{8292183}将信号功率谱密度图作为图像输入CNN,成功对WiFi、蓝牙等多种无线技术进行了分类,并初步探讨了重叠信号的分类问题,展示了深度学习在复杂电磁环境下的应用潜力。
尽管如此,这类方法依旧沿袭了“先分割、后分类”的思路,未能解决信号在时间和频率上的精确定位问题。

综上所述,在传统信号处理框架下,信号的定位与分类始终是两条相互独立的技术路径。直至计算机视觉领域的目标检测技术取得长足发展,才为实现端到端的时频联合定位与分类提供了全新的可能。

\subsection{基于深度学习的目标检测方法研究现状}
深度学习的浪潮极大地推动了计算机视觉领域的发展,其中核心任务之一即为目标检测,旨在从图像中同时预测出目标的边界框(Bounding Box)并对目标进行分类。

在目标检测领域研究初期,以Faster R-CNN\cite{7485869}为代表的两阶段(two-stage)检测器取得了巨大成功。
该方法首先通过区域提议网络(Rigion Proposal Network, RPN)生成一系列可能包含目标的候选框(anchor),然后再对这些候选框进行二次分类和位置精修,精度较高但速度较慢。
与之相对的是以YOLO\cite{redmon2016you}和SSD\cite{liu2016ssd}为代表的单阶段(one-stage)检测器,它们取消了区域提议步骤,直接在特征图上预测目标的类别和位置,实现了更快的检测速度。
其中,YOLO系列是典型的无锚框(anchor-free)算法,而SSD则沿用了锚框(anchor-based)机制。

一个关键的里程碑是特征金字塔网络(Feature Pyramid Network, FPN)\cite{8099589}的提出。
FPN通过构建多尺度的特征图,并融合高层语义信息和底层位置信息,显著提升了模型对不同尺度目标(尤其是小目标)的检测能力。
此后,引入FPN或其变体已成为现代检测器的标准配置。
在FPN的基础上,RetinaNet\cite{8237586}通过引入Focal Loss解决了单阶段检测器中正负样本极度不平衡的问题,在保持高速的同时达到了媲美两阶段检测器的精度;
FCOS\cite{9010746}则是一种更为彻底的anchor-free单阶段检测器,它将目标检测视为一种逐像素的预测任务,进一步简化了模型设计。

近年来,在自然语言处理领域取得成功的Transformer\cite{vaswani2017attention}架构也被逐渐引入到计算机视觉中。
DETR\cite{carion2020end}是首个将Transformer完全应用于目标检测的工作,它将目标检测视为一个集合预测问题,无需非极大值抑制(Non-Maximum Suppression, NMS)等后处理操作,实现了端到端的检测。
但早期的DETR存在收敛速度慢、对小目标检测能力不足等问题,且未有效利用FPN的多尺度思想。
为了解决这些问题,Deformable DETR\cite{zhu2020deformable}应运而生,它通过引入可变形注意力机制,并结合多尺度特征,显著提升了模型的性能和收敛速度。

计算机视觉中目标检测技术的演进,为解决信号时频联合检测这一难题提供了直接的理论借鉴与丰富的模型储备。

\subsection{基于深度学习的信号检测方法研究现状}
借鉴于目标检测领域的成功,研究者们开始将相关框架应用于二维时频谱图的分析,以实现信号的时频定位与分类识别的一体化处理。

早期的探索性工作验证了这一思路的可行性。
Ke等人\cite{8753512}将CNN与长短期记忆网络(Long-Short Term Memory, LSTM)相结合,以分别提取信号的频率特征和时间(空间)特征,对非合作通信信号进行盲检测。
然而,该研究并未直接采用成熟的目标检测框架。随后,主流的目标检测模型被陆续引入。
Zha等人\cite{zha2019deep}采用了SSD框架进行信号检测与调制分类,
Prasad等人\cite{9128779}将Faster R-CNN框架用于宽带系统中的盲时频定位,并在同年\cite{9075413}对其进行了轻量化裁剪;
Li等人\cite{9408998}则采用了改进的YOLOv3模型进行信号的盲检测。
这些工作证明了将经典目标检测模型迁移至信号领域的有效性,但也暴露出共同的局限:
(1)所用数据集多为仿真生成,信号种类有限(例如仅包含WiFi信号或仅分为连续/突发两类),且未考虑信号重叠等复杂情况,与真实的复杂电磁环境存在显著脱节。
(2)这些研究更侧重于方案的可行性验证,在模型结构创新和性能提升方面探索不足。

为提升模型对特定信号形态的检测能力,后续研究开始探索更具针对性的网络结构。
Li等人\cite{9788039}针对高频宽带频谱图,提出了一种无锚框的框架。该框架利用每个频率点下的所有时间点特征来联合预测信号的中心频率和形状属性。
这项工作在数据集的复杂度上有所提升(单个样本中包含了更多的信号实例),但其设计的网络结构较为简单,完全基于卷积,缺乏对长距离依赖的建模能力,且其逐频点预测的机制使其天然无法处理频率上重叠的信号。

最新的研究则开始关注更具挑战性的场景和更先进的网络结构。
Peng等人\cite{10640092}提出了一种结合残差扩张网络和水平移位注意力的模型,其数据集包含了拥挤的信号环境,且在模型的性能上也有所提升。
然而,该研究仍存在明显不足:
(1)信号类别和调制方式依然有限(仅分为两类信号和三种调制方式);
(2)模型仍局限于CNN架构,未能引入表现出强大建模能力的Transformer;
(3)忽略了在真实测量场景中,由于采样率和采样时间不同导致STFT时频谱图尺寸不统一的实际问题。

综上所述,将深度学习目标检测应用于信号时频联合检测已成为该领域的主流趋势,并展现出巨大潜力。
然而,现有研究仍存在以下主要不足:
(1)大多数研究使用的数据集或为仿真生成,或在信号类型、密度、信噪比范围以及重叠复杂性上与真实电磁环境存在较大差距,限制了模型的泛化能力和实用价值;
(2)多数研究直接套用计算机视觉领域的成熟模型,缺乏对信号时频谱图特有物理属性(如信号的连续性、时变性、长宽比悬殊等)的针对性设计和优化,尤其在长距离依赖关系建模和重叠信号处理上能力欠缺;
(3)没有考虑在真实测量场景中,由于采样率和采样时间不同导致STFT时频谱图尺寸不统一的实际问题。
因此,本课题拟围绕上述不足展开研究,旨在设计一种能够应对复杂真实电磁环境、具备更强泛化能力与工程实用性的新型深度学习模型,以解决信号时频联合检测与分类的挑战。
