% !TEX root = ../main.tex

\chapter{课题背景及研究的目的与意义}

% ========== 1.1 课题背景 ==========
\section{课题背景}

随着物联网(Internet of Things, IoT)技术的迅猛发展,无线通信设备数量正呈爆炸式增长。
从智能家居、工业自动化到智慧医疗与无人驾驶,海量的无线设备与通信技术在有限的频谱资源中交织共存,构成了日益拥挤且动态多变的复杂电磁环境。
在此背景下,信号检测(Signal Detection)技术的重要性日益凸显,它不仅是提升频谱共享效率、降低信号间干扰的关键,也是保障无线通信安全、监控非法信号的必要手段。

与仅关注频段是否被占用的传统频谱感知(Spectrum Sensing, SS)不同,
信号检测旨在同时完成两项核心子任务:
(1)对环境中的信号进行精确的时频定位(即确定信号在何时占用哪些频谱资源);
(2)对定位到的信号进行准确的制式识别(即判断信号的调制类型或通信协议)。
由此可知,信号检测是实现动态频谱接入(Dynamic Spectrum Access, DSA)和保障通信可靠性的关键前提。

然而,能量检测、匹配滤波的传统频段定位方式和基于机器学习的传统信号分类方式由于其对噪声功率敏感、依赖信号先验信息以及计算复杂度高等固有局限,
难以完全适应当前非协作、高动态的复杂电磁环境。

近年来,深度学习(Deep Learning, DL)领域的突破性进展为应对上述挑战提供了全新思路。
通过将一维时域信号经过短时傅里叶变换(Short-Time Fourier Transform, STFT)转化为二维时频图,
信号感知任务得以巧妙地转化为计算机视觉领域中技术成熟的目标检测(Object Detection)问题。
这种跨域迁移方法的优势在于,
它能够借助目标检测模型实现端到端的处理,在无需精确信号或信道先验知识的条件下,自动学习信号在时频域的深层特征,从而在信号的精确定位与制式分类方面展现出巨大潜力。

% ========== 1.2 研究的目的与意义 ==========
\section{研究的目的与意义}

尽管计算机视觉领域的目标检测技术已日趋成熟,并涌现出众多先进方案,
但其在信号检测领域的应用仍处于初步探索阶段,不仅相关研究成果有限,且现有的技术迁移模式也存在若干亟待解决的问题:

1. 数据集与真实场景存在偏差:
目前,该领域的研究大多采用仿真生成或混合真实信号的数据集。
这些数据集普遍存在对现实场景的过度简化,例如信号制式与调制类型有限、信号时频密度较低,且通常忽略了信号间的重叠与干扰。
此外,现有数据集普遍假设了固定的采样率与采样时长,这与实际监测中参数多变的情况存在显著差异。
虽然这种简化有助于在研究初期快速验证算法的可行性,但却使其难以准确评估模型在真实复杂电磁环境下的性能与泛化能力。

2. 模型迁移缺乏针对性优化:
现有的研究在模型迁移方面,大多直接套用计算机视觉领域的经典模型(如Faster R-CNN),而未根据信号时频图的内在特征(相较于自然图像形态差异巨大)对模型结构进行针对性调整。
虽然近期有研究开始尝试对模型进行改进,但仍主要局限于传统的卷积神经网络(Convolutional Neural Network, CNN)架构,未能引入近年来在计算机视觉领域取得突破性进展的Transformer架构(如DETR、Deformable DETR等),其潜力有待进一步发掘。

因此,本课题旨在针对上述问题展开深入研究。研究将在技术迁移的基础上,充分考虑更为复杂多变的信号检测场景,从数据处理、模型结构设计到训练策略等环节进行系统性的优化。
其最终目标是实现目标检测技术与信号检测任务的深度融合,显著提升信号检测模型的准确率与鲁棒性。
这一技术在具体实践中将直接赋能两大关键应用:
(1)在DSA中提高信号对总体时频资源的利用率,同时降低信号之间的干扰;
(2)在复杂电磁环境中识别已授权和未授权的信号,并精确定位非法信号,从而保障无线通信的电磁空间安全。
